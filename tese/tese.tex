% Arquivo LaTeX de exemplo de dissertação/tese a ser apresentada à CPG do IME-USP
%
% Criação: Jesús P. Mena-Chalco
% Revisão: Fabio Kon e Paulo Feofiloff
% Adaptação para UTF8, biblatex e outras melhorias: Nelson Lago
%
% Except where otherwise indicated, these files are distributed under
% the MIT Licence. The example text, which includes the tutorial and
% examples as well as the explanatory comments in the source, are
% available under the Creative Commons Attribution International
% Licence, v4.0 (CC-BY 4.0) - https://creativecommons.org/licenses/by/4.0/


%%%%%%%%%%%%%%%%%%%%%%%%%%%%%%%%%%%%%%%%%%%%%%%%%%%%%%%%%%%%%%%%%%%%%%%%%%%%%%%%
%%%%%%%%%%%%%%%%%%%%%%%%%%%%%%% PREÂMBULO LaTeX %%%%%%%%%%%%%%%%%%%%%%%%%%%%%%%%
%%%%%%%%%%%%%%%%%%%%%%%%%%%%%%%%%%%%%%%%%%%%%%%%%%%%%%%%%%%%%%%%%%%%%%%%%%%%%%%%

% A opção twoside (frente-e-verso) significa que a aparência das páginas pares
% e ímpares pode ser diferente. Por exemplo, as margens podem ser diferentes ou
% os números de página podem aparecer à direita ou à esquerda alternadamente.
% Mas nada impede que você crie um documento "só frente" e, ao imprimir, faça
% a impressão frente-e-verso.
%
% Aqui também definimos a língua padrão do documento (a última da lista) e
% línguas adicionais. Para teses do IME, no mínimo português e inglês são
% obrigatórios, porque independentemente da língua principal do texto é
% preciso fornecer o resumo nessas duas línguas. LaTeX aceita alguns nomes
% diferentes para a língua portuguesa; dentre as opções, prefira sempre
% "brazilian" para português brasileiro e "portuguese" para português europeu.
%\documentclass[a4paper,12pt,twoside,brazilian,english]{book}
\documentclass[a4paper,12pt,twoside,english,brazilian]{book}

% O preâmbulo de um documento LaTeX pode ser razoavelmente longo. Neste
% modelo, optamos por reduzi-lo, colocando praticamente tudo do preâmbulo
% nas packages "imegoodies" e "imelooks".
%
% imegoodies carrega diversas packages muito úteis e populares (algumas
% são praticamente obrigatórias, como amsmath, babel, array etc.). É
% uma boa ideia usá-la com outros documentos também. Ela inclui vários
% comentários explicativos e dicas de uso; não tenha medo de alterá-la
% conforme a necessidade.
%
% imelooks carrega algumas packages e configurações que definem a
% aparência do documento; você também pode querer usá-la (ou partes
% dela) com outros documentos para obter as mesmas fontes, margens
% etc. Tal como "imegoodies", pode valer a pena ler os comentários
% e fazer modificações nessa package. Com a opção "thesis", imelooks
% também define os comandos para capa, folha de rosto etc.
\usepackage{configuracao/imegoodies}
\usepackage[thesis]{configuracao/imelooks}

%\nocolorlinks % para impressão em P&B

% Diretórios onde estão as figuras; com isso, não é necessário (mas
% é permitido) colocar o caminho completo em \includegraphics. Note
% que a extensão nunca é necessária (mas é permitida), ou seja, o
% resultado é o mesmo com "\includegraphics{figuras/foto.jpeg}",
% "\includegraphics{foto.jpeg}", "\includegraphics{figuras/foto}"
% ou "\includegraphics{foto}".
\graphicspath{{figuras/},{fig/},{logos/},{img/},{images/},{imagens/}}

% Comandos rápidos para mudar de língua:
% \en -> muda para o inglês
% \br -> muda para o português
% \texten{blah} -> o texto "blah" é em inglês
% \textbr{blah} -> o texto "blah" é em português
\babeltags{br = brazilian, en = english}


%%%%%%%%%%%%%%%%%%%%%%%%%%%%%%%%%%%%%%%%%%%%%%%%%%%%%%%%%%%%%%%%%%%%%%%%%%%%%%%%
%%%%%%%%%%%%%%%%%%%%%%%%%%%%%%%%%% METADADOS %%%%%%%%%%%%%%%%%%%%%%%%%%%%%%%%%%%
%%%%%%%%%%%%%%%%%%%%%%%%%%%%%%%%%%%%%%%%%%%%%%%%%%%%%%%%%%%%%%%%%%%%%%%%%%%%%%%%

% O arquivo com os dados bibliográficos para biblatex; você pode usar
% este comando mais de uma vez para acrescentar múltiplos arquivos
\addbibresource{bibliografia.bib}

% Este comando permite acrescentar itens à lista de referências sem incluir
% uma referência de fato no texto (pode ser usado em qualquer lugar do texto)
%\nocite{bronevetsky02,schmidt03:MSc, FSF:GNU-GPL, CORBA:spec, MenaChalco08}
% Com este comando, todos os itens do arquivo .bib são incluídos na lista
% de referências
%\nocite{*}

% É possível definir como determinadas palavras podem (ou não) ser
% hifenizadas; no entanto, a hifenização automática geralmente funciona bem
\babelhyphenation{documentclass latexmk soft-ware clsguide} % todas as línguas
\babelhyphenation[brazilian]{Fu-la-no}
\babelhyphenation[english]{what-ever}

% Estes comandos definem o título e autoria do trabalho e devem sempre ser
% definidos, pois além de serem utilizados para criar a capa, também são
% armazenados nos metadados do PDF. O subtítulo é opcional.
\title{A IA generativa na Engenharia de Software}[Um estudo de caso]
\translatedtitle{Generative AI in Software Engineering}[A case study]

\author{Cássio Azevedo Cancio}

\def\profa{Prof\kern.02em.\kern-.07emª\kern.07em}
\def\dra{Dr\kern-.04em.\kern-.11emª\kern.07em}

% Para TCCs, este comando define o supervisor
\orientador{Prof. Dr. Paulo Roberto Miranda Meirelles}

% Se não houver, remova; se houver mais de um, basta
% repetir o comando quantas vezes forem necessárias
\coorientador{Arthur Pilone Maia da Silva}
\coorientador{Carlos Eduardo Santos}

% TODO - remover esse trecho?
\banca{
  \profa{} \dra{} Fulana de Tal (orientadora) -- IME-USP [sem ponto final],
  % Em inglês, não há o "ª"
  %Prof. Dr. Fulana de Tal (advisor) -- IME-USP [sem ponto final],
  Prof. Dr. Ciclano de Tal -- IME-USP [sem ponto final],
  \profa{} \dra{} Convidada de Tal -- IMPA [sem ponto final],
}

% A página de rosto da versão para depósito (ou seja, a versão final
% antes da defesa) deve ser diferente da página de rosto da versão
% definitiva (ou seja, a versão final após a incorporação das sugestões
% da banca).
\tipotese{
  %mestrado,
  %doutorado,
  tcc,
  %definitiva, % É a versão para defesa ou a versão definitiva?
  %quali, % É qualificação?
  programa={Ciência da Computação},
}
% TODO - remover esse trecho?
\defesa{
  local={São Paulo},
  data=2025-06-17, % YYYY-MM-DD
}

% A licença do seu trabalho. Use CC-BY, CC-BY-NC, CC-BY-ND, CC-BY-SA,
% CC-BY-NC-SA ou CC-BY-NC-ND para escolher a licença Creative Commons
% correspondente (o sistema insere automaticamente o texto da licença).
% Se quiser estabelecer regras diferentes para o uso de seu trabalho,
% converse com seu orientador e coloque o texto da licença aqui, mas
% observe que apenas TCCs sob alguma licença Creative Commons serão
% acrescentados ao BDTA. Se você tem alguma intenção de publicar o
% trabalho comercialmente no futuro, sugerimos a licença CC-BY-NC-ND.
%
%\direitos{CC-BY-NC-ND}
%
%\direitos{Autorizo a reprodução e divulgação total ou parcial deste
%          trabalho, por qualquer meio convencional ou eletrônico,
%          para fins de estudo e pesquisa, desde que citada a fonte.}
%
%\direitos{I authorize the complete or partial reproduction and disclosure
%          of this work by any conventional or electronic means for study
%          and research purposes, provided that the source is acknowledged.}
%
\direitos{CC-BY}

% Para gerar a ficha catalográfica, acesse https://fc.ime.usp.br/,
% preencha o formulário e escolha a opção "Gerar Código LaTeX".
% Basta copiar e colar o resultado aqui.
\fichacatalografica{}


%%%%%%%%%%%%%%%%%%%%%%%%%%%%%%%%%%%%%%%%%%%%%%%%%%%%%%%%%%%%%%%%%%%%%%%%%%%%%%%%
%%%%%%%%%%%%%%%%%%%%%%% AQUI COMEÇA O CONTEÚDO DE FATO %%%%%%%%%%%%%%%%%%%%%%%%%
%%%%%%%%%%%%%%%%%%%%%%%%%%%%%%%%%%%%%%%%%%%%%%%%%%%%%%%%%%%%%%%%%%%%%%%%%%%%%%%%

\begin{document}

%%%%%%%%%%%%%%%%%%%%%%%%%%% CAPA E PÁGINAS INICIAIS %%%%%%%%%%%%%%%%%%%%%%%%%%%%

% Aqui começa o conteúdo inicial que aparece antes do capítulo 1, ou seja,
% página de rosto, resumo, sumário etc. O comando frontmatter faz números
% de página aparecem em algarismos romanos ao invés de arábicos e
% desabilita a contagem de capítulos.
\frontmatter

\pagestyle{plain}

\onehalfspacing % Espaçamento 1,5 na capa e páginas iniciais

\maketitle % capa e folha de rosto

%%%%%%%%%%%%%%%% DEDICATÓRIA, AGRADECIMENTOS, RESUMO/ABSTRACT %%%%%%%%%%%%%%%%%%

\begin{dedicatoria}
Aos meus pais, que sempre incentivaram meus estudos.
Aos meus professores, que tornaram este trabalho possível.
\end{dedicatoria}

% Reinicia o contador de páginas (a próxima página recebe o número "i") para
% que a página da dedicatória não seja contada.
\pagenumbering{roman}

% Agradecimentos:
% Se o candidato não quer fazer agradecimentos, deve simplesmente eliminar
% esta página. A epígrafe, obviamente, é opcional; é possível colocar
% epígrafes em todos os capítulos. O comando "\chapter*" faz esta seção
% não ser incluída no sumário.
% \chapter*{Agradecimentos}
% \epigrafe{Do. Or do not. There is no try.}{Mestre Yoda}

% Texto texto texto texto texto texto texto texto texto texto texto texto texto
% texto texto texto texto texto texto texto texto texto texto texto texto texto
% texto texto texto texto texto texto texto texto texto texto texto texto texto
% texto texto texto texto. Texto opcional.

%!TeX root=../tese.tex
%("dica" para o editor de texto: este arquivo é parte de um documento maior)
% para saber mais: https://tex.stackexchange.com/q/78101

% As palavras-chave são obrigatórias, em português e em inglês, e devem ser
% definidas antes do resumo/abstract. Acrescente quantas forem necessárias.
\palavraschave{Palavra-chave1, Palavra-chave2, Palavra-chave3}

\keywords{Keyword1,Keyword2,Keyword3}

% O resumo é obrigatório, em português e inglês. Estes comandos também
% geram automaticamente a referência para o próprio documento, conforme
% as normas sugeridas da USP.
% TODO - escrever o resumo
\resumo{
Elemento obrigatório, constituído de uma sequência de frases concisas e
objetivas, em forma de texto. Deve apresentar os objetivos, métodos empregados,
resultados e conclusões. O resumo deve ser redigido em parágrafo único, conter
no máximo 500 palavras e ser seguido dos termos representativos do conteúdo do
trabalho (palavras-chave). Deve ser precedido da referência do documento.
Texto texto texto texto texto texto texto texto texto texto texto texto texto
texto texto texto texto texto texto texto texto texto texto texto texto texto
texto texto texto texto texto texto texto texto texto texto texto texto texto
texto texto texto texto texto texto texto texto texto texto texto texto texto
texto texto texto texto texto texto texto texto texto texto texto texto texto
texto texto texto texto texto texto texto texto.
Texto texto texto texto texto texto texto texto texto texto texto texto texto
texto texto texto texto texto texto texto texto texto texto texto texto texto
texto texto texto texto texto texto texto texto texto texto texto texto texto
texto texto texto texto texto texto texto texto texto texto texto texto texto
texto texto.
}

\abstract{
Elemento obrigatório, elaborado com as mesmas características do resumo em
língua portuguesa. De acordo com o Regimento da Pós-Graduação da USP (Artigo
99), deve ser redigido em inglês para fins de divulgação. É uma boa ideia usar
o sítio \url{www.grammarly.com} na preparação de textos em inglês.
Text text text text text text text text text text text text text text text text
text text text text text text text text text text text text text text text text
text text text text text text text text text text text text text text text text
text text text text text text text text text text text text.
Text text text text text text text text text text text text text text text text
text text text text text text text text text text text text text text text text
text text text.
}



%%%%%%%%%%%%%%%%%%%%%%%%%%% LISTAS DE FIGURAS ETC. %%%%%%%%%%%%%%%%%%%%%%%%%%%%%

% Como as listas que se seguem podem não incluir uma quebra de página
% obrigatória, inserimos uma quebra manualmente aqui.
\cleardoublepage

% Todas as listas são opcionais; Usando "\chapter*" elas não são incluídas
% no sumário. As listas geradas automaticamente também não são incluídas por
% conta das opções "notlot" e "notlof" que usamos para a package tocbibind.

% Normalmente, "\chapter*" faz o novo capítulo iniciar em uma nova página, e as
% listas geradas automaticamente também por padrão ficam em páginas separadas.
% Como cada uma destas listas é muito curta, não faz muito sentido fazer isso
% aqui, então usamos este comando para desabilitar essas quebras de página.
% Se você preferir, comente as linhas com esse comando e des-comente as linhas
% sem ele para criar as listas em páginas separadas. Observe que você também
% pode inserir quebras de página manualmente (com \clearpage, veja o exemplo
% mais abaixo).
\newcommand\disablenewpage[1]{{\let\clearpage\par\let\cleardoublepage\par #1}}

% Nestas listas, é melhor usar "raggedbottom" (veja basics.tex). Colocamos
% a opção correspondente e as listas dentro de um grupo para ativar
% raggedbottom apenas temporariamente.
\bgroup
\raggedbottom

%%%%% Listas criadas manualmente

%\chapter*{Lista de abreviaturas}
\disablenewpage{\chapter*{Lista de abreviaturas}}

\begin{tabular}{rl}
   IA & Inteligência Artificial (\emph{Artificial Intelligence})\\
   IME & Instituto de Matemática e Estatística\\
   LLM & Modelo de Linguagem de Grande Escala (\emph{Large Language Model})\\
   USP & Universidade de São Paulo
\end{tabular}

%\chapter*{Lista de símbolos}
% \disablenewpage{\chapter*{Lista de símbolos}}

% \begin{tabular}{rl}
%   $\omega$ & Frequência angular\\
%     $\psi$ & Função de análise \emph{wavelet}\\
%     $\Psi$ & Transformada de Fourier de $\psi$\\
% \end{tabular}

% Quebra de página manual
\clearpage

%%%%% Listas criadas automaticamente

% Você pode escolher se quer ou não permitir a quebra de página
\listoffigures
% \disablenewpage{\listoffigures}

% Você pode escolher se quer ou não permitir a quebra de página
\listoftables
% \disablenewpage{\listoftables}

% Esta lista é criada "automaticamente" pela package float quando
% definimos o novo tipo de float "program" (em utils.tex)
% Você pode escolher se quer ou não permitir a quebra de página
\listof{program}{\programlistname}
% \disablenewpage{\listof{program}{\programlistname}}

% Sumário (obrigatório)
\tableofcontents

\egroup % Final de "raggedbottom"

% Referências indiretas ("x", veja "y") para o índice remissivo (opcionais,
% pois o índice é opcional). É comum colocar esses itens no final do documento,
% junto com o comando \printindex, mas em alguns casos isso torna necessário
% executar texindy (ou makeindex) mais de uma vez, então colocar aqui é melhor.
% TODO - mudar esses índices remissivos
\index{Inglês|see{Língua estrangeira}}
\index{Figuras|see{Floats}}
\index{Tabelas|see{Floats}}
\index{Código-fonte|see{Floats}}
\index{Subcaptions|see{Subfiguras}}
\index{Sublegendas|see{Subfiguras}}
\index{Equações|see{Modo matemático}}
\index{Fórmulas|see{Modo matemático}}
\index{Rodapé, notas|see{Notas de rodapé}}
\index{Captions|see{Legendas}}
\index{Versão original|see{Tese/Dissertação, versões}}
\index{Versão corrigida|see{Tese/Dissertação, versões}}
\index{Palavras estrangeiras|see{Língua estrangeira}}
\index{Floats!Algoritmo|see{Floats, ordem}}


%%%%%%%%%%%%%%%%%%%%%%%%%%%%%%%% CAPÍTULOS %%%%%%%%%%%%%%%%%%%%%%%%%%%%%%%%%%%%%

% Aqui vai o conteúdo principal do trabalho, ou seja, os capítulos que compõem
% a dissertação/tese. O comando mainmatter reinicia a contagem de páginas,
% modifica a numeração para números arábicos e ativa a contagem de capítulos.
\mainmatter

\pagestyle{mainmatter}

% Espaçamento simples
\singlespacing

% A introdução não tem número de capítulo, então os cabeçalhos também não
\pagestyle{unnumberedchapter}
%!TeX root=../tese.tex
%("dica" para o editor de texto: este arquivo é parte de um documento maior)
% para saber mais: https://tex.stackexchange.com/q/78101

%% ------------------------------------------------------------------------- %%

% "\chapter" cria um capítulo com número e o coloca no sumário; "\chapter*"
% cria um capítulo sem número e não o coloca no sumário. A introdução não
% deve ser numerada, mas deve aparecer no sumário. Por conta disso, este
% modelo define o comando "\chapter**".
\chapter**{Introdução}\label{cap:introducao}

\section**{Contexto}

A engenharia de \emph{software} é um campo da computação que surgiu e se desenvolveu através da crescente demanda da sociedade do fim do século XX até a atualidade por sistemas computacionais cada vez mais complexos. Neste contexto, ferramentas, métodos e processos foram criados para possibilitar o atendimento dessa demanda.

Nas últimas décadas, os estudos em inteligência artificial (IA) avançaram rapidamente, de modo que um novo paradigma em IA surgiu, a IA generativa. Diferentemente da IA tradicional, a IA generativa pode criar conteúdos novos e originais baseados no que aprendeu, em vez de apenas copiar, imitar e reproduzir algo que já existe.

Dada a flexibilidade e a abertura de diversas possibilidades com essa nova tecnologia, é natural que uma de suas aplicações fosse a engenharia de \emph{software}. Nos últimos anos, diversos estudos foram publicados de modo a analisar essas aplicações, suas consequências e propor diferentes abordagens para tais aplicações. (~\cite{10705752} e~\cite{10.1145/3715003})

Desta forma, este projeto se propõe a levantar os resultados observados em diversos artigos que tratam do estudo do impacto da IA generativa na engenharia de \emph{software}. Além de realizar um estudo de caso sobre a aplicação da IA generativa ao longo das fases do desenvolvimento de \emph{software}, avaliando sua utilidade, limitações e impacto na qualidade e produtividade.

Como estudo de caso, foi realizado o desenvolvimento de uma aplicação web na área de investimentos, com banco de dados, \emph{backend} em \emph{Java Spring Boot} e \emph{frontend} em \emph{Angular}, utilizando ferramentas de IA generativa nas diferentes fases do desenvolvimento de um sistema. As fases analisadas foram: coleta e análise de requisitos, estudo de viabilidade, \emph{design} de \emph{software}, codificação e testes.

\section**{Motivação}

Este trabalho se faz relevante no contexto em que o uso de ferramentas de IA generativa vem crescendo com o passar dos anos, desde o surgimento de ferramentas como \emph{ChatGPT} e \emph{GitHub Copilot}. Segundo dados da \emph{Stack Overflow 2024 Developer Survey}~\citep{StackOverflow2024}, 63,2\% dos desenvolvedores profissionais já utilizam ferramentas de IA no seu processo de desenvolvimento, enquanto que 13.5\% desse mesmo grupo planeja utilizá-las em breve. Além disso, entre os desenvolvedores que responderam usar inteligência artificial, 82\% a utiliza para escrever código.

Desta maneira, é evidente que uma nova tecnologia com amplo uso no mercado de \emph{software} e que abre possibilidade para diversas aplicações, terá impactos sobre como os desenvolvedores escrevem seus códigos. Assim, é de suma importância buscar avaliar e compreender melhor de que maneira esses impactos vêm ocorrendo nas bases de código, inclusive através de um estudo de caso.

\section**{Objetivos}

Os principais objetivos do trabalho são:

\begin{itemize}
  \item Compreender de que maneira a IA generativa tem impactado na produção de código das empresas de \emph{software}, através de um levantamento de dados disponíveis em outros artigos;
  \item Desenvolver um sistema com todo seu processo voltado ao uso de ferramentas de IA generativa ao longo das suas diferentes fases;
  \item Documentar os resultados gerados pela IA durante o processo de desenvolvimento do sistema, incluindo os \emph{prompts} utilizados;
  \item Analisar os resultados obtidos, a fim de mensurar a qualidade das respostas geradas.
\end{itemize}
\pagestyle{mainmatter}
% !TeX root=../tese.tex
%("dica" para o editor de texto: este arquivo é parte de um documento maior)
% para saber mais: https://tex.stackexchange.com/q/78101

%% ------------------------------------------------------------------------- %%

\chapter{Referencial Teórico}\label{cap:fundamentacao}

\section{Engenharia de Software}
Segundo a definição do \emph{Institute of Electrical and Electronics Engineers} (IEEE) (\citeyear{159342}), a engenharia de \emph{software} é a aplicação de uma sistemática, disciplinada e quantificável abordagem para o desenvolvimento, operação e manutenção de um \emph{software}. Para que a engenharia de \emph{software} seja viável, dada a complexidade dos sistemas demandados na atualidade, foram desenvolvidas etapas, metodologias e ferramentas que dessem suporte aos atores envolvidos no projeto, como desenvolvedores, analistas, investidores, clientes, entre outros.

\subsection{Etapas do Desenvolvimento de Software}
O Ciclo de Vida de Desenvolvimento de Software (SDLC) consiste numa sequência de processos pelos quais o desenvolvimento de um \emph{software} ocorre, de modo a produzir um resultado eficaz e de alta qualidade. Existe alguma variação no número de passos descritos por diferentes fontes, mas, em geral, há sete fases essenciais: planejamento, análise de requisitos, \emph{design}, codificação, testes, implantação e manutenção.

\subsubsection{Planejamento}

A fase inicial envolve definir o propósito e o escopo do \emph{software}. Durante esta etapa, a equipe de desenvolvimento deve levantar as tarefas necessárias, elaborar estratégias para cumpri-las e colaborar de modo a compreender as necessidades dos usuários finais. Neste processo, os objetivos do \emph{software} e qual problema ele se propõe a resolver precisam ficar claros a todos os envolvidos.

Além disso, nesta fase também ocorre o estudo de viabilidade, ou seja, desenvolvedores e outros atores do projeto avaliam desafios técnicos e financeiros que possam impactar a evolução ou o sucesso do \emph{software}. Ao fim desta fase, um plano de projeto é criado, com o intuito de detalhar as funções do sistema, os recursos necessários, possíveis riscos e o cronograma do projeto. Ao definir papéis, responsabilidades e expectativas claras, o planejamento estabelece uma base sólida para um processo eficiente de desenvolvimento.

\subsubsection{Análise de Requisitos}

A segunda fase do SDLC visa identificar e registrar os requisitos dos usuários finais. Nesta etapa, a equipe de projeto realiza o levantamento dos requisitos, por meio da coleta de informações das partes interessadas, como analistas, usuários e clientes. São empregadas técnicas como entrevistas, pesquisas e grupos de foco para compreender as necessidades e expectativas dos usuários.

Após a coleta, os dados são analisados, diferenciando os requisitos essenciais dos desejáveis. Essa análise possibilita a definição das funcionalidades, desempenho, segurança e interfaces do \emph{software}. Neste momento, são definidos os requisitos funcionais e não funcionais. Os requisitos funcionais especificam as funções que o \emph{software} deve realizar, ou seja, o que o sistema deve fazer. Já os requisitos não funcionais tratam de como o sistema deve se comportar, incluindo aspectos como desempenho, segurança, usabilidade e escalabilidade.

O resultado desse processo é o Documento de Especificação de Requisitos (DER), que descreve o propósito, as funcionalidades e características do \emph{software}, servindo como guia para a equipe de desenvolvimento e fornecendo estimativas de custo, quando necessário. O êxito desta fase é crucial para o sucesso do projeto, pois assegura que a solução desenvolvida atenda às expectativas dos usuários.

\subsubsection{Design}

A fase de \emph{design} é responsável pela definição da estrutura do \emph{software}, abrangendo sua funcionalidade e aparência. A equipe de desenvolvimento detalha a arquitetura do sistema, a navegação, as interfaces de usuário e a modelagem do banco de dados, assegurando que o \emph{software} tenha boa usabilidade e seja eficiente.

Entre as atividades desta fase, destaca-se a elaboração de diagramas de fluxo de dados, de entidade-relacionamento, de classes, protótipos de interface e diagramas arquiteturais. O objetivo é garantir que as estruturas projetadas sejam suficientes para dar suporte a todas as funcionalidades do sistema. Também são identificadas dependências, pontos de integração e eventuais restrições, como limitações do equipamento físico e requisitos de desempenho.

O resultado desta fase é o Documento de \emph{Design} de Software (DDS) que estrutura formalmente as informações do projeto e trata preocupações de \emph{design}. Neste documento, são adicionados os artefatos produzidos, servindo como guia estável para coordenar equipes grandes e garantir que todos os componentes do sistema funcionem de maneira integrada.

\subsubsection{Codificação}

Na fase de codificação, os engenheiros e desenvolvedores transformam o \emph{design} do \emph{software} em código executável. O objetivo é produzir um \emph{software} funcional, eficiente e com boa usabilidade. Para isso, utilizam-se linguagens de programação adequadas, seguindo o DDS e diretrizes de codificação estabelecidas pela organização e pela legislação local.

Durante esta fase, são realizadas revisões de código, nas quais os membros da equipe examinam o trabalho uns dos outros para identificar erros ou inconsistências, garantindo elevados padrões de qualidade. Além disso, testes preliminares internos são conduzidos para garantir que as funcionalidades básicas do sistema foram atendidas.

Ao final da fase de codificação, o \emph{software} passa a existir como um produto funcional, representando a materialização dos esforços das etapas anteriores, mesmo que ainda sejam necessários refinamentos e ajustes subsequentes. O resultado desta fase é o código-fonte.

\subsubsection{Testes}

A fase de testes consiste em verificar a qualidade e a confiabilidade do \emph{software} antes de sua entrega aos usuários finais. Seu objetivo é identificar falhas, erros e vulnerabilidades, assegurando que o sistema atenda aos requisitos especificados.

Inicialmente, são definidos parâmetros de teste alinhados aos requisitos do \emph{software} e casos de teste que contemplem diferentes cenários de uso. Em seguida, são conduzidos testes de diversos níveis e tipos, incluindo testes de unidade, de integração, de sistema, de segurança e de aceitação, permitindo a avaliação tanto de componentes individuais quanto da operação do sistema na sua totalidade.

Quando um erro é identificado, ele é registrado detalhadamente, incluindo seu comportamento, métodos de reprodução e impacto sobre o sistema. As falhas são encaminhadas para correção e o \emph{software} retorna à fase de testes para validação. Este ciclo de teste e correção se repete até que o sistema esteja conforme os critérios previamente estabelecidos. O resultado desta fase é um código-fonte mais robusto e menos propenso a falhas.

\subsubsection{Implantação}

A fase de implantação ou \emph{deployment} consiste em disponibilizar o \emph{software} aos usuários finais, garantindo sua operacionalidade no ambiente de produção. Este processo ocorre tanto no primeiro lançamento do sistema, quanto quando ele já está em uso pelos usuários e passando por atualizações. Por isso, o processo deve minimizar interrupções e impactos negativos nos acessos dos usuários.

A escolha da estratégia de implantação é feita conforme as características do sistema e de seus usuários. As estratégias mais comuns são:

\begin{itemize}
    \item \textbf{\emph{Rolling}:} a atualização ocorre de forma gradual, substituindo instâncias antigas por novas até que todo o sistema esteja atualizado;
    \item \textbf{\emph{Blue-Green}:} dois ambientes paralelos são mantidos, um em produção e outro em preparação, permitindo a troca imediata entre eles;
    \item \textbf{\emph{Canary}:} a nova versão é liberada primeiramente para um grupo de usuários, monitorando o comportamento do sistema antes de expandir a implantação para todos.
\end{itemize}

Além de colocar o \emph{software} em operação, esta fase envolve assegurar que os usuários compreendam seu funcionamento. Para isso, podem ser fornecidos manuais, treinamentos e suporte técnico. Desta maneira, a fase de implantação marca a transição do \emph{software} de projeto para produto, iniciando efetivamente o cumprimento de seus objetivos e a entrega de valor ao usuário.

\subsubsection{Manutenção}

A fase de manutenção é caracterizada por suporte contínuo e por melhorias incrementais, de modo a garantir que o \emph{software} mantenha seu funcionamento adequado, acompanhe as necessidades dos usuários e as demandas de mercado. Nesta fase, são realizadas atualizações, correções de falhas e suporte ao usuário.

Considerando o horizonte de longo prazo, a manutenção inclui estratégias de modernização ou substituição do \emph{software}, buscando manter sua relevância e adequação às evoluções tecnológicas.

\subsection{Metodologias de Desenvolvimento}

As metodologias de desenvolvimento de \emph{software} consistem em abordagens sistemáticas para organizar, planejar e executar projetos de engenharia de \emph{software}. Cada metodologia apresenta vantagens e limitações, sendo a escolha dependente do tamanho do projeto, grau de complexidade, maturidade da equipe e expectativa de mudanças nos requisitos.

\subsubsection{Método de Cascata}

A metodologia cascata é uma abordagem sequencial em que cada fase do desenvolvimento de \emph{software} deve ser concluída antes de iniciar a seguinte. Ela é adequada para projetos com requisitos muito definidos e pouca probabilidade de mudanças durante o desenvolvimento. O modelo enfatiza documentação detalhada e planejamento prévio, garantindo controle rígido sobre prazos e entregas. Embora simples de aplicar, pode se mostrar inflexível diante de alterações nos requisitos ou no ambiente de negócios.

\subsubsection{Metodologias Ágeis}

As metodologias ágeis consistem em práticas iterativas e incrementais, voltadas à entrega contínua de valor ao usuário e à adaptação rápida às mudanças nos requisitos. As abordagens ágeis valorizam colaboração, flexibilidade e melhoria contínua, sendo especialmente adequada para projetos dinâmicos e de alta complexidade.

Além disso, elas são geralmente complementadas por práticas de Integração Contínua e Entrega Contínua (CI/CD), que automatizam a construção, teste e implantação do \emph{software}. Desta maneira, a utilização de CI/CD permite que novas funcionalidades e correções sejam disponibilizadas rapidamente, mantendo a qualidade do sistema e reduzindo o tempo de reposta entre a equipe de desenvolvimento e os usuários.

Alguns exemplos de métodos ágeis incluem:
\begin{itemize}
    \item \textbf{\emph{Scrum}:} método iterativo que organiza o trabalho em ciclos curtos, as iterações (\emph{sprints}), com duração de uma a quatro semanas. Define papéis específicos, como o responsável pelo produto (\emph{product owner}), o facilitador do processo (\emph{scrum master}) e a equipe de desenvolvimento. Inclui ainda reuniões regulares, como o planejamento da iteração (\emph{sprint planning}), reuniões diárias de acompanhamento (\emph{daily}), a revisão da iteração e a retrospectiva, que garantem planejamento, monitoramento e melhoria contínua do processo;

    \item \textbf{\emph{Kanban}:} método visual de gestão do fluxo de trabalho, baseado na utilização de quadros divididos em colunas que representam etapas do processo (por exemplo, ``A Fazer'', ``Em Progresso'' e ``Concluído''). Cada atividade é representada por um cartão que se movimenta conforme avança nas etapas, promovendo transparência, fluxo contínuo e foco na identificação e eliminação de gargalos;

    \item \textbf{Programação extrema (\emph{extreme programming}):} metodologia ágil que enfatiza práticas de desenvolvimento colaborativo e de qualidade, como programação em par, integração contínua, desenvolvimento orientado a testes e refatoração frequente. Valoriza o envolvimento próximo do cliente, a simplicidade do código e a adaptação rápida às mudanças de requisitos;

    \item \textbf{Desenvolvimento enxuto (\emph{lean software development}):} abordagem inspirada nos princípios da produção enxuta, que busca eliminar desperdícios, reduzir custos e acelerar entregas. Essa metodologia prioriza a criação de valor para o cliente, a melhoria contínua e a capacitação das equipes para tomadas de decisão mais eficientes.
\end{itemize}

\subsection{Ferramentas de Desenvolvimento}

As ferramentas de desenvolvimento de software oferecem suporte às etapas do ciclo de vida, desde planejamento até manutenção. O uso combinado dessas ferramentas contribui para maior produtividade, qualidade e confiabilidade no desenvolvimento de software. Elas incluem:

\begin{itemize}
    \item \textbf{Sistemas de Controle de Versão (como \emph{Git} e \emph{SVN}):} permitem registrar e gerenciar alterações no código-fonte temporalmente, possibilitando a colaboração simultânea entre desenvolvedores, a recuperação de versões anteriores e o rastreamento completo do histórico de mudanças;

    \item \textbf{Ambientes de Desenvolvimento Integrados (IDEs) (como \emph{Visual Studio}, \emph{IntelliJ IDEA} e \emph{Eclipse}):} oferecem um conjunto de ferramentas em um único ambiente, incluindo edição de código, depuração, testes, gerenciamento de dependências e integração com sistemas de controle de versão;

    \item \textbf{Ferramentas de Gerenciamento de Projetos (como \emph{Jira}, \emph{Trello} e \emph{Asana}):} auxiliam na organização e priorização de tarefas, acompanhamento do progresso e comunicação entre membros da equipe, fornecendo transparência e facilitando a coordenação do trabalho;

    \item \textbf{Ferramentas de Integração e Entrega Contínua (CI/CD) (como \emph{Jenkins}, \emph{GitHub Actions} e \emph{GitLab CI}):} automatizam processos de compilação, testes e implantação, promovendo maior qualidade e agilidade nas entregas de software;

    \item \textbf{Ferramentas de Teste (como \emph{Selenium}, \emph{JUnit} e \emph{Postman}):} permitem a execução de testes automatizados e manuais para validar funcionalidades, desempenho e segurança do sistema, contribuindo para a detecção precoce de falhas e a melhoria da qualidade do software.
\end{itemize}

\section{Inteligência Artificial}
\subsection{Conceitos Básicos}

Inteligência Artificial (IA) é o campo da ciência da computação que se dedica a criar sistemas capazes de executar tarefas que normalmente exigem inteligência humana, como reconhecimento de padrões, tomada de decisão e aprendizado a partir de dados. A IA pode ser classificada em:

\begin{itemize}
    \item \textbf{IA Fraca (Narrow AI)}: sistemas projetados para executar tarefas específicas, sem consciência ou compreensão geral do mundo.
    \item \textbf{IA Forte (General AI)}: hipotéticos sistemas capazes de realizar qualquer tarefa cognitiva que um humano consegue executar.
    \item \textbf{IA Baseada em Aprendizado de Máquina}: sistemas que melhoram seu desempenho por meio da análise de grandes volumes de dados e ajuste de parâmetros internos.
\end{itemize}

\subsection{Redes Neurais}

Redes neurais artificiais são modelos computacionais inspirados na estrutura do cérebro humano. Elas são compostas por camadas de nós (neurônios artificiais) interconectados, capazes de processar informações e aprender padrões a partir de exemplos. Redes neurais são amplamente utilizadas em tarefas como classificação, reconhecimento de imagens e processamento de linguagem natural.

\subsection{Aprendizado de Máquina}

Aprendizado de Máquina (Machine Learning) é um subcampo da IA que se concentra em criar algoritmos capazes de aprender e fazer previsões a partir de dados, sem serem explicitamente programados para cada tarefa. Os principais tipos de aprendizado incluem:

\begin{itemize}
    \item \textbf{Supervisionado}: o algoritmo aprende a partir de exemplos rotulados, tentando prever saídas corretas para novas entradas.
    \item \textbf{Não Supervisionado}: o algoritmo identifica padrões ou estruturas em dados não rotulados.
    \item \textbf{Por Reforço}: o algoritmo aprende tomando ações em um ambiente e recebendo recompensas ou penalidades, buscando maximizar o desempenho ao longo do tempo.
\end{itemize}

\section{IA Generativa}
\subsection{Modelos de Linguagem}

Modelos de linguagem são algoritmos capazes de compreender, gerar e manipular texto em linguagem natural. Eles aprendem padrões estatísticos e semânticos a partir de grandes corpora de texto e são a base para sistemas de tradução automática, assistentes virtuais e chatbots avançados.

\subsection{Transformers}

Transformers são uma arquitetura de redes neurais introduzida em 2017, projetada para lidar eficientemente com sequências de dados, como texto. Utilizam mecanismos de atenção que permitem processar relações entre palavras de forma paralela, superando limitações de modelos recorrentes tradicionais.

\subsection{LLMs (\emph{Large Language Models})}

Large Language Models (LLMs) são modelos de linguagem treinados com enormes volumes de dados textuais, contendo bilhões de parâmetros. Eles são capazes de gerar texto coerente, resumir informações, responder perguntas e realizar tarefas complexas de processamento de linguagem natural. Exemplos incluem GPT, BERT e T5.

%!TeX root=../tese.tex
%("dica" para o editor de texto: este arquivo é parte de um documento maior)
% para saber mais: https://tex.stackexchange.com/q/78101

% Vamos definir alguns comandos auxiliares para facilitar.

% "textbackslash" é muito comprido.
\newcommand{\sla}{\textbackslash}

% Vamos escrever comandos (como "make" ou "itemize") com formatação especial.
\newcommand{\cmd}[1]{\textsf{#1}}

% Idem para packages; aqui estamos usando a mesma formatação de \cmd,
% mas poderíamos escolher outra.
\newcommand{\pkg}[1]{\textsf{#1}}

% A maioria dos comandos LaTeX começa com "\"; vamos criar um
% comando que já coloca essa barra e formata com "\cmd".
\newcommand{\ltxcmd}[1]{\cmd{\sla{}#1}}

\chapter{Metodologia}
\label{cap:metodologia}

\section{Abordagem de Pesquisa}
\subsection{Tipo de Pesquisa}
\subsection{Procedimentos Metodológicos}

\section{Coleta de Dados}
\subsection{Fontes de Dados}
\subsection{Instrumentos de Coleta}
\subsection{Processo de Coleta}

\section{Análise de Dados}
\subsection{Métodos de Análise}
\subsection{Ferramentas Utilizadas}
\subsection{Critérios de Avaliação}

%!TeX root=../tese.tex
%("dica" para o editor de texto: este arquivo é parte de um documento maior)
% para saber mais: https://tex.stackexchange.com/q/78101

%% ------------------------------------------------------------------------- %%

\chapter{Resultados}
\label{cap:resultados}

\section{Análise dos Dados}

\section{Avaliação do Sistema}
\subsection{Desempenho}
\subsection{Eficiência}
\subsection{Usabilidade}

\section{Discussão}
\subsection{Limitações Identificadas}
\subsection{Melhorias Propostas}
%!TeX root=../tese.tex
%("dica" para o editor de texto: este arquivo é parte de um documento maior)
% para saber mais: https://tex.stackexchange.com/q/78101

%% ------------------------------------------------------------------------- %%

\chapter{Conclusão}
\label{cap:conclusao}

\section{Resumo dos Resultados}
\subsection{Principais Descobertas}
\subsection{Objetivos Alcançados}
\subsection{Contribuições}

\section{Trabalhos Futuros}
\subsection{Direções de Pesquisa}
\subsection{Melhorias Propostas}
\subsection{Desafios Identificados}

\section{Considerações Finais}


%%%%%%%%%%%%%%%%%%%%%%%%%%%% APÊNDICES E ANEXOS %%%%%%%%%%%%%%%%%%%%%%%%%%%%%%%%

% Um apêndice é algum conteúdo adicional de sua autoria que faz parte e
% colabora com a ideia geral do texto mas que, por alguma razão, não precisa
% fazer parte da sequência do discurso; por exemplo, a demonstração de um
% teorema intermediário, as perguntas usadas em uma pesquisa qualitativa etc.
%
% Um anexo é um documento que não faz parte da tese (em geral, nem é de sua
% autoria) mas é relevante para o conteúdo; por exemplo, a especificação do
% padrão técnico ou a legislação que o trabalho discute, um artigo de jornal
% apresentando a percepção do público sobre o tema da tese etc.
%
% Os comandos appendix e annex reiniciam a numeração de capítulos e passam
% a numerá-los com letras. "annex" não faz parte de nenhuma classe padrão,
% foi criado para este modelo. Se o trabalho não tiver apêndices ou anexos,
% remova estas linhas.
%
% Diferentemente de \mainmatter, \backmatter etc., \appendix e \annex não
% forçam o início de uma nova página. Em geral isso não é importante, pois
% o comando seguinte costuma ser "\chapter", mas pode causar problemas com
% a formatação dos cabeçalhos. Assim, vamos forçar uma nova página antes
% de cada um deles.

%%%% Apêndices %%%%

\cleardoublepage

\pagestyle{appendix}

\appendix

% \addappheadtotoc acrescenta a palavra "Apêndice" ao sumário; se
% só há apêndices, sem anexos, provavelmente não é necessário.
\addappheadtotoc

% %!TeX root=../tese.tex
%("dica" para o editor de texto: este arquivo é parte de um documento maior)
% para saber mais: https://tex.stackexchange.com/q/78101

\chapter{Perguntas frequentes sobre o modelo}

\begin{itemize}

\item \textbf{Não consigo decorar tantos comandos!}\\
Use a colinha que é distribuída juntamente com este modelo (\url{gitlab.com/ccsl-usp/modelo-latex/raw/main/pre-compilados/colinha.pdf?inline=false}).

\item \textbf{Estou tendo problemas com caracteres acentuados.}\\
Versões modernas de \LaTeX{} usam UTF-8, mas arquivos antigos podem usar outras codificações (como ISO-8859-1, também conhecido como latin1 ou Windows-1252). Nesses casos, use \textsf{\textbackslash{}usepackage[latin1]\{inputenc\}} no preâmbulo do documento. Você também pode representar os caracteres acentuados usando comandos \LaTeX{}: \textsf{\textbackslash\textquotesingle{}a} para á, \textsf{\textbackslash{}c\{c\}} para cedilha etc., independentemente da codificação usada no texto\footnote{Você pode consultar os comandos desse tipo mais comuns em \url{en.wikibooks.org/wiki/LaTeX/Special_Characters}. Observe que a dica sobre o pingo do i \emph{não} é mais válida atualmente; basta usar \textsf{\textbackslash\textquotesingle{}i}.}.

\item \textbf{É possível resumir o nome das seções/capítulos que aparece no topo das páginas e no sumário?}\\
Sim, usando a sintaxe \textsf{\textbackslash{}section[mini-titulo]\{titulo enorme\}}. Isso é especialmente útil nas legendas (\textit{captions}\index{Legendas}) das figuras e tabelas, que muitas vezes são demasiadamente longas para a lista de figuras/tabelas.

\item \textbf{Existe algum programa para gerenciar referências em formato bibtex?}\\
Sim, há vários. Uma opção bem comum é o JabRef; outra é usar Zotero\index{Zotero} ou Mendeley\index{Mendeley} e exportar os dados deles no formato .bib.

\item \textbf{Posso usar pacotes \LaTeX{} adicionais aos sugeridos?}\\
Com certeza! Você pode modificar os arquivos o quanto desejar, o modelo serve só como uma ajuda inicial para o seu trabalho.

\end{itemize}

\par

%%%% Anexos %%%%

\cleardoublepage

\pagestyle{appendix} % repete o anterior, caso você não use apêndices

\annex

% \addappheadtotoc acrescenta a palavra "Anexo" ao sumário; se
% só há anexos, sem apêndices, provavelmente não é necessário.
\addappheadtotoc

% %!TeX root=../tese.tex
%("dica" para o editor de texto: este arquivo é parte de um documento maior)
% para saber mais: https://tex.stackexchange.com/q/78101

\chapter{As packages \pkg{imegoodies} e \pkg{imelooks}}
\label{ann:imegoodlooks}

Este modelo inclui as \textit{packages} \pkg{imegoodies} e \pkg{imelooks},
que você pode querer usar em outros documentos \LaTeX.

\pkg{imegoodies} inclui um grande número de \textit{packages} que são
comumente usadas e bastante úteis. Em geral, você pode incluí-la em seus
documentos sem que isso cause problemas de compatibilidade. Se, no
entanto, algo não funcionar, você pode editar o arquivo para eliminar
a \textit{package} responsável pelo problema se ela não for necessária.
\pkg{imegoodies} ainda inclui vários comentários explicativos sobre as
\textit{packages} carregadas.

\pkg{imelooks} também inclui um grande número de \textit{packages}, mas
estas são relacionadas mais explicitamente à aparência do documento
(fontes, cores, margens etc.). Você também pode utilizá-la em outros
documentos se quiser se aproximar da aparência deste modelo. \pkg{imelooks}
reconhece diversos parâmetros que ativam/desativam aspectos específicos:

\begin{itemize}
  \item \cmd{fonts} carrega as fontes deste modelo (libertinus e
        sourcecodepro), além de outros pequenos ajustes relacionados.
        Esta opção é sempre ativada por padrão; para desativá-la, use
        \cmd{nofonts}

  \item \cmd{spacing} utiliza os espaçamentos definidos neste modelo (margens,
        espaço entre parágrafos, indentação da primeira linha do parágrafo
        etc.). Esta opção é sempre ativada por padrão; para desativá-la, use
        \cmd{nospacing}

  \item \cmd{captions} e \cmd{footnotes} fazem respectivamente as legendas
        (das figuras e tabelas) e as notas de rodapé de acordo com este modelo.
        Estas opções são sempre ativadas por padrão; para desativá-las, use
        \cmd{nocaptions} e \cmd{nofootnotes}

  \item \cmd{autohttp} acrescenta o prefixo \cmd{http://} a URLs criadas
        com \ltxcmd{url} que não incluam o \textit{schema}. Esta opção é
        sempre ativada por padrão; para desativá-la, use \cmd{noautohttp}

  \item \cmd{hidelinks}, \cmd{borderlinks} e \cmd{colorlinks} definem a
        aparência dos hiperlinks. \cmd{hidelinks} faz os hiperlinks sem
        nenhuma formatação especial; \cmd{borderlinks} faz os hiperlinks
        serem envidos por um quadrado colorido (apenas na tela; o quadrado
        não é impresso); \cmd{colorlinks} faz o texto dos hiperlinks ser
        colorido. A opção \cmd{colorlinks} é sempre ativada por padrão

  \item \cmd{biblatex} carrega a \textit{package} \cmd{biblatex} e os
        estilos bibliográficos deste modelo. Esta opção é sempre ativada
        por padrão; para desativá-la, use \cmd{nobiblatex}
  \item \cmd{raggedbib} faz a bibliografia (com \cmd{biblatex}) ser
        formatada com alinhamento à esquerda ao invés de justificado.
        Esta opção é sempre ativada por padrão, exceto quando o estilo
        bibliográfico é \cmd{plainnat-ime} (usado nas teses); para
        desativá-la, use \cmd{noraggedbib}; para ativá-la incondicionalmente,
        use \cmd{raggedbib}
  \item \cmd{bibstyle=?} selectiona um estilo bibliográfico específico.
        O estilo padrão é \cmd{numeric}, exceto em pôsteres e apresentações
        (\cmd{beamer-ime}) e \textit{reports} (\cmd{plainnat-ime})

  \item \cmd{listings} carrega a \textit{package} \cmd{listings} e diversas
        configurações relacionadas usadas neste modelo. Esta opção é
        sempre ativada por padrão; para desativá-la, use \cmd{nolistings}

  \item \cmd{greeny}, \cmd{bluey}, \cmd{sandy} ativam esquemas de cores
        diferentes para pôsteres e apresentações (o padrão é \cmd{bluey})

  \item \cmd{beamer} \textbf{des}ativa algumas \textit{packages} que
        são incompatíveis com a classe \cmd{beamer} (note que as opções
        \cmd{slides} e \cmd{presentation}, discutidas abaixo, já fazem isso)

  \item \cmd{presentation} (ou \cmd{slides}) e \cmd{poster} ativam as
        opções relevantes para, respectivamente, apresentações com
        \cmd{beamer} ou pôsteres com \cmd{tcolorbox}

  \item \cmd{report} ativa as opções relevantes para documentos com
        capítulos (cabeçalhos das páginas, características do sumário etc.)

  \item \cmd{thesis} ativa a opção \cmd{report} e também define o que é
        necessário para a geração da capa das teses de acordo com este modelo

  \item \cmd{resumoabstract} define os comandos \cmd{resumo} e \cmd{abstract}
        de acordo com este modelo. Esta opção é ativada por padrão com
        \cmd{report}; para desativá-la, use \cmd{noresumoabstract}

  \item \cmd{brazilian} verifica se a língua portuguesa está ativa no
        documento e, em caso negativo, gera um erro. Esta opção é
        ativada por padrão com a opção \cmd{thesis}; para desativá-la,
        use \cmd{nobrazilian}
\end{itemize}

\par
% %!TeX root=../tese.tex
%("dica" para o editor de texto: este arquivo é parte de um documento maior)
% para saber mais: https://tex.stackexchange.com/q/78101

\chapter{Código-fonte e pseudocódigo}
\label{ap:pseudocode}

Com a \textit{package} \textsf{listings}, programas podem ser inseridos
diretamente no arquivo, como feito no caso do Programa~\ref{prog:java},
ou importados de um arquivo externo com o comando
\textsf{\textbackslash{}lstinputlisting}, como no caso
do Programa~\ref{prog:mdcinput}.

% O exemplo foi copiado da documentação de algorithmicx
\begin{program}
  \lstinputlisting[
    language=pseudocode,
    style=pseudocode,
    style=wider,
    functions={},
    specialidentifiers={},
  ]
  {template/euclid.psc}

  \caption{Máximo divisor comum (arquivo importado).\label{prog:mdcinput}}
\end{program}

Trechos de código curtos (menores que uma página) podem ou não ser
incluídos como \textit{floats}; trechos longos necessariamente incluem
quebras de página e, portanto, não podem ser \textit{floats}. Com
\textit{floats}, a legenda e as linhas separadoras são colocadas pelo
comando \textsf{\textbackslash{}begin\{program\}}; sem eles, utilize o
ambiente \textsf{programruledcaption} (atenção para a colocação do
comando \textsf{\textbackslash{}label\{\}}, dentro da legenda), como
no Programa~\ref{prog:mdc}\footnote{\textsf{listings} oferece alguns
recursos próprios para a definição de \textit{floats} e legendas, mas
neste modelo não os utilizamos.}:

\begin{programruledcaption}{Máximo divisor comum (em português).\label{prog:mdc}}
  \begin{lstlisting}[
    language={[brazilian]pseudocode},
    style=pseudocode,
    style=wider,
    functions={},
    specialidentifiers={},
  ]
      funcao euclides(a, b) // O máximo divisor comum de \textbf{a} e \textbf{b}
          r := a $\bmod$ b
	  enquanto r != 0 // Atingimos a resposta se \textbf{r} é zero
              a := b
              b := r
              r := a $\bmod$ b
          fim
	  devolva b // O máximo divisor comum é \textbf{b}
      fim
  \end{lstlisting}
\end{programruledcaption}

Além do suporte às várias linguagens incluídas em \textsf{listings},
este modelo traz uma extensão para permitir o uso de pseudocódigo,
útil para a descrição de algoritmos em alto nível. Ela oferece
diversos recursos:

\begin{itemize}

    \item Comentários seguem o padrão de C++ (\lstinline{//} e
          \lstinline{/* ... */}), mas o delimitador é impresso
          como ``$\triangleright$''.

    \item ``:='', ``<>'', ``<='', ``>='' e ``!='' são substituídos
          pelo símbolo matemático adequado.

    \item É possível acrescentar palavras-chave além de ``if'', ``and''
          etc. com a opção ``\textsf{morekeywords=\{pchave1,\linebreak[0]{}pchave2\}}''
          (para um trecho de código específico) ou com o comando
          \textsf{\textbackslash{}lstset\{morekeywords=\linebreak[0]{}\{pchave1,pchave2\}\}}
          (como comando de configuração geral).

    \item É possível usar pequenos trechos de código, como nomes de variáveis,
          dentro de um parágrafo normal com \textsf{\textbackslash{}lstinline\{blah\}}.

    \item ``\$\dots\$'' ativa o modo matemático em qualquer lugar.

    \item Outros comandos \LaTeX{} funcionam apenas em comentários; fora, a
          linguagem simula alguns pré-definidos (\textsf{\textbackslash{}textit\{\}},
          \textsf{\textbackslash{}texttt\{\}} etc.).

    \item O comando \textsf{\textbackslash{}label} também funciona em
          comentários; a referência correspondente (\textsf{\textbackslash{}ref})
          indica o número da linha de código. Se quiser usá-lo numa linha sem
          comentários, use \lstinline{///}~\textsf{\textbackslash{}label\{blah\}};
          ``\lstinline{///}'' funciona como \lstinline{//}, permitindo
          a inserção de comandos \LaTeX{}, mas não imprime o delimitador
          (\ensuremath{\triangleright}).

    \item Para suspender a formatação automática, use \textsf{\textbackslash{}noparse\{blah\}}.

    \item Para forçar a formatação de um texto como função, identificador,
          palavra-chave ou comentário, use \textsf{\textbackslash{}func\{blah\}},
          \textsf{\textbackslash{}id\{blah\}}, \textsf{\textbackslash{}kw\{blah\}} ou
          \textsf{\textbackslash{}comment\{blah\}}.

    \item Palavras-chave dentro de comentários não são formatadas
          automaticamente; se necessário, use \textsf{\textbackslash{}func\{\}},
          \textsf{\textbackslash{}id\{\}} etc. ou comandos \LaTeX{} padrão.

    \item As palavras ``Program'', ``Procedure'' e ``Function'' têm formatação
          especial e fazem a palavra seguinte ser formatada como função.
          Funções em outros lugares \emph{não} são detectadas automaticamente;
          use \textsf{\textbackslash{}func\{\}}, a opção ``\textsf{functions=\{func1,func2\}}''
          ou o comando ``\textsf{\textbackslash{}lstset\{functions=\{func1,func2\}\}}''
          para que elas sejam detectadas.

    \item Além de funções, palavras-chave, strings, comentários e
          identificadores, há ``\textsf{specialidentifiers}''. Você pode
          usá-los com \textsf{\textbackslash{}specialid\{blah\}}, com a opção
          ``\textsf{specialidentifiers=\{id1,id2\}}'' ou com o comando
          ``\textsf{\textbackslash{}lstset\{specialidentifiers=\{id1,id2\}\}}''.

\end{itemize}



\par


%%%%%%%%%%%%%%% SEÇÕES FINAIS (BIBLIOGRAFIA E ÍNDICE REMISSIVO) %%%%%%%%%%%%%%%%

% O comando backmatter desabilita a numeração de capítulos.
\backmatter

\pagestyle{backmatter}

% Espaço adicional no sumário antes das referências / índice remissivo
\addtocontents{toc}{\vspace{2\baselineskip plus .5\baselineskip minus .5\baselineskip}}

% A bibliografia é obrigatória

\printbibliography[
  title=\refname\label{sec:bib}, % "Referências", recomendado pela ABNT
  %title=\bibname\label{sec:bib}, % "Bibliografia"
  heading=bibintoc, % Inclui a bibliografia no sumário
]

\printindex % imprime o índice remissivo no documento (opcional)

\end{document}
