%!TeX root=../tese.tex

\chapter{Conclusão}
\label{cap:conclusao}

3.5. Limitações da metodologia

Apesar de ampla, a metodologia apresenta limitações, entre as quais:
	•	Dependência da literatura cinzenta, que pode conter vieses institucionais, comerciais ou metodológicos.
	•	Falta de padronização entre reports, que utilizam diferentes métodos de coleta, tamanhos de amostra e categorias de análise.
	•	Foco em documentos de grande impacto, o que pode excluir estudos menores ou alternativos que ofereçam visões diferentes.
	•	Uso de um único artigo principal da literatura formal, o que, embora permita uma análise estruturada, reduz o espectro acadêmico considerado.

Essas limitações são discutidas ao final do trabalho, juntamente com sugestões de pesquisas futuras.




3.6. Possíveis continuidades

A partir da metodologia adotada, diversas linhas de continuidade são possíveis:
	•	expansão da análise para mais artigos acadêmicos e estudos longitudinais;
	•	avaliação empírica do uso de ferramentas de IA em ambientes reais de desenvolvimento;
	•	investigação da adoção de LLMs em contextos educacionais;
	•	estudo da relação entre qualidade do código gerado e maturidade das ferramentas;
	•	incorporação de dados de novos relatórios da indústria à medida que forem publicados.


