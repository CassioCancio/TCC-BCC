% !TeX root=../tese.tex
%("dica" para o editor de texto: este arquivo é parte de um documento maior)
% para saber mais: https://tex.stackexchange.com/q/78101

%% ------------------------------------------------------------------------- %%

\chapter{Referencial Teórico}\label{cap:fundamentacao}

\section{Engenharia de Software}
Segundo a definição do \citet{159342}, a engenharia de \emph{software} é a aplicação de uma sistemática, disciplinada e quantificável abordagem para o desenvolvimento, operação e manutenção de um \emph{software}. Para que a engenharia de \emph{software} seja viável, dada a complexidade dos sistemas demandados na atualidade, foram desenvolvidas etapas, metodologias e ferramentas que dessem suporte aos atores envolvidos no projeto, como desenvolvedores, analistas, investidores, clientes, entre outros.

\subsection{Etapas do Desenvolvimento de Software}
O Ciclo de Vida de Desenvolvimento de Software (SDLC) consiste numa sequência de processos pelos quais o desenvolvimento de um \emph{software} ocorre, de modo a produzir um resultado eficaz e de alta qualidade. Existe alguma variação no número de passos descritos por diferentes fontes, mas, em geral, há sete fases essenciais: planejamento, análise de requisitos, \emph{design}, codificação, testes, implantação e manutenção.

\subsubsection{Planejamento}

A fase inicial envolve definir o propósito e o escopo do \emph{software}. Durante esta etapa, a equipe de desenvolvimento deve levantar as tarefas necessárias, elaborar estratégias para cumpri-las e colaborar de modo a compreender as necessidades dos usuários finais. Neste processo, os objetivos do \emph{software} precisam ficar claros a todos os envolvidos.

Além disso, nesta fase ocorre o estudo de viabilidade, ou seja, desenvolvedores e outros atores do projeto avaliam desafios técnicos e financeiros que possam impactar a evolução e o sucesso do sistema. Ao fim desta fase, um plano de projeto é criado, com o intuito de detalhar as funções do sistema, os recursos necessários, possíveis riscos e o cronograma de execução. Ao definir papéis, responsabilidades e expectativas claras, o planejamento estabelece uma base sólida para um processo eficiente de desenvolvimento.

\subsubsection{Análise de Requisitos}

Nesta etapa, a equipe de projeto realiza o levantamento dos requisitos, por meio da coleta de informações das partes interessadas, como analistas, usuários e clientes. São empregadas técnicas como entrevistas, pesquisas e grupos de foco para compreender as necessidades e expectativas dos usuários.

Após a coleta, os dados são analisados, diferenciando os requisitos essenciais dos desejáveis. Essa análise possibilita a definição das funcionalidades, desempenho, segurança e interfaces do \emph{software}. Neste momento, são definidos os requisitos funcionais e não funcionais. Os requisitos funcionais especificam as funções que o \emph{software} deve ter, já os requisitos não funcionais tratam de como o sistema deve se comportar, incluindo aspectos como desempenho, segurança, usabilidade e escalabilidade.

O resultado desse processo é o Documento de Especificação de Requisitos (DER), que descreve o propósito, as funcionalidades e características do \emph{software}, servindo como guia para a equipe de desenvolvimento e fornecendo estimativas de custo. O êxito desta fase é crucial para o sucesso do projeto, pois assegura que a solução desenvolvida atenda às expectativas dos usuários.

\subsubsection{Design}

A fase de \emph{design} é responsável pela definição da estrutura do \emph{software}, abrangendo sua funcionalidade e aparência. A equipe de desenvolvimento detalha a arquitetura do sistema, a navegação, as interfaces de usuário e a modelagem do banco de dados, assegurando que o \emph{software} tenha boa usabilidade e seja eficiente.

Entre as atividades desta fase, destaca-se a elaboração de diagramas de fluxo de dados, de entidade-relacionamento, de classes, protótipos de interface e diagramas arquiteturais. O objetivo é garantir que as estruturas projetadas sejam suficientes para dar suporte a todas as funcionalidades do sistema. Também são identificadas dependências, pontos de integração e eventuais restrições, como limitações do equipamento físico e requisitos de desempenho.

O resultado desta fase é o Documento de \emph{Design} de Software (DDS) que estrutura formalmente as informações do projeto e trata preocupações de \emph{design}. Neste documento, são adicionados os artefatos produzidos, servindo como guia estável para coordenar equipes grandes e garantir que todos os componentes do sistema funcionem de maneira integrada.

\subsubsection{Codificação}

Na fase de codificação, os engenheiros e desenvolvedores transformam o \emph{design} do \emph{software} em código executável. O objetivo é produzir um \emph{software} funcional, eficiente e com boa usabilidade. Para isso, são utilizadas linguagens de programação adequadas, seguindo o DDS e diretrizes de codificação estabelecidas pela organização e pela legislação local.

Durante esta fase, são realizadas revisões de código, nas quais os membros da equipe examinam o trabalho uns dos outros para identificar erros ou inconsistências, garantindo elevados padrões de qualidade. Além disso, testes preliminares internos são conduzidos para garantir que as funcionalidades básicas do sistema sejam atendidas.

Ao final da fase de codificação, o \emph{software} passa a existir como um produto funcional, representando a materialização dos esforços das etapas anteriores, mesmo que ainda sejam necessários refinamentos e ajustes subsequentes. O resultado desta fase é o código-fonte.

\subsubsection{Testes}

A fase de testes consiste em verificar a qualidade e a confiabilidade do \emph{software} antes de sua entrega aos usuários finais. Seu objetivo é identificar falhas, erros e vulnerabilidades, assegurando que o sistema atenda aos requisitos especificados.

Inicialmente, são definidos parâmetros de teste alinhados aos requisitos do \emph{software} e casos de teste que contemplem diferentes cenários de uso. Em seguida, são conduzidos testes de diversos níveis e tipos, incluindo testes de unidade, de integração, de sistema, de segurança e de aceitação, permitindo a avaliação tanto de componentes individuais quanto da operação do sistema na sua totalidade.

Quando um erro é identificado, ele é registrado detalhadamente, incluindo seu comportamento, métodos de reprodução e impacto sobre o sistema. As falhas são encaminhadas para correção e o \emph{software} retorna à fase de testes para validação. Este ciclo de teste e correção se repete até que o sistema esteja conforme os critérios previamente estabelecidos. O resultado desta fase é um código-fonte mais robusto e menos propenso a falhas.

\subsubsection{Implantação}

A fase de implantação ou \emph{deployment} consiste em disponibilizar o \emph{software} aos usuários finais, garantindo sua operacionalidade no ambiente de produção. Este processo ocorre tanto no primeiro lançamento do sistema, quanto quando ele já está em uso pelos usuários e passando por atualizações, neste caso, o processo deve minimizar interrupções e impactos no acesso dos usuários.

Além de colocar o \emph{software} em operação, esta fase envolve assegurar que os usuários compreendam seu funcionamento. Para isso, podem ser fornecidos manuais, treinamentos e suporte técnico. Desta maneira, a fase de implantação marca a transição do \emph{software} de projeto para produto, iniciando efetivamente o cumprimento de seus objetivos e a entrega de valor ao usuário.

\subsubsection{Manutenção}

A fase de manutenção é caracterizada por suporte contínuo e por melhorias incrementais, de modo a garantir que o \emph{software} mantenha seu funcionamento adequado, acompanhe as necessidades dos usuários e as demandas de mercado. Nesta fase, são realizadas atualizações, correções de falhas e suporte ao usuário. Considerando o horizonte de longo prazo, a manutenção inclui estratégias de modernização ou substituição do \emph{software}, buscando manter sua relevância e adequação às evoluções tecnológicas.

\subsection{Ferramentas de Desenvolvimento}

As ferramentas de desenvolvimento de \emph{software} oferecem suporte às etapas do SDLC. O uso combinado dessas ferramentas contribui para maior produtividade, qualidade e confiabilidade no desenvolvimento de \emph{software}. Elas incluem ferramentas de:

\begin{itemize}
    \item \textbf{Controle de Versão (como \emph{Git} e \emph{SVN}):} permitem registrar e gerenciar alterações no código-fonte temporalmente, possibilitando a colaboração simultânea entre desenvolvedores, a recuperação de versões anteriores e o rastreamento completo do histórico de mudanças;

    \item \textbf{Ambientes de Desenvolvimento Integrados (IDEs) (como \emph{Visual Studio}, \emph{IntelliJ IDEA} e \emph{Eclipse}):} oferecem um conjunto de ferramentas em um único ambiente, incluindo edição de código, depuração, testes, gerenciamento de dependências, integração com sistemas de controle de versão e ferramentas de IA generativa;

    \item \textbf{Gerenciamento de Projetos (como \emph{Jira}, \emph{Trello} e \emph{Asana}):} auxiliam na organização e priorização de tarefas, acompanhamento do progresso e comunicação entre membros da equipe, fornecendo transparência e facilitando a coordenação do trabalho;

    \item \textbf{Integração e Entrega Contínua (CI/CD) (como \emph{Jenkins}, \emph{GitHub Actions} e \emph{GitLab CI}):} automatizam processos de compilação, testes e implantação, promovendo maior qualidade e agilidade nas entregas de \emph{software};

    \item \textbf{Teste (como \emph{Selenium}, \emph{JUnit} e \emph{Postman}):} permitem a execução de testes automatizados e manuais para validar funcionalidades, desempenho e segurança do sistema, contribuindo para a detecção precoce de falhas e a melhoria da qualidade do \emph{software}.

    \item \textbf{Virtualização e Monitoramento (como \emph{Docker}, \emph{Prometheus} e \emph{Grafana}):} permitem criar ambientes isolados e consistentes para execução do \emph{software}, garantindo que ele funcione de maneira idêntica em diferentes máquinas. Além disso, possibilitam acompanhar o desempenho e a saúde dos sistemas em produção, auxiliando na detecção precoce de problemas na qualidade do \emph{software}.
\end{itemize}

\section{Inteligência Artificial}
Inteligência Artificial (IA) é o campo da ciência da computação que se dedica a criar sistemas capazes de executar tarefas que normalmente exigiriam inteligência humana, como reconhecimento de padrões, raciocínio, tomada de decisão, resolução de problemas e aprendizado a partir de dados. Segundo uma declaração da IEEE \citep{ieee2019ai}, a inteligência artificial inclui tecnologias computacionais inspiradas no modo como pessoas e outros organismos biológicos percebem, aprendem, raciocinam e agem.

As aplicações de IA afetam cada vez mais diversos aspectos da sociedade, incluindo defesa e segurança nacional, sistemas de justiça, comércio, finanças, manufatura, saúde, transporte, educação, entretenimento e interações sociais. Aplicações como essas estão se expandindo pela combinação de processadores avançados, grandes volumes de dados e novos algoritmos. Segundo um estudo da \citet{mckinsey2018-ai-impact}, a IA contribuirá com cerca de 13 trilhões de dólares para o PIB global até 2030.

\subsection{Aprendizado de Máquina}

Aprendizado de Máquina (\emph{Machine Learning}) é um subcampo da IA que se concentra em criar algoritmos capazes de aprender e fazer previsões a partir de quantidades expressivas de dados, geralmente estruturados ou rotulados, sem serem explicitamente programados para cada tarefa. Um conceito importante para o aprendizado de máquina são as redes neurais, modelos computacionais inspirados na estrutura do cérebro humano, compostas por camadas de nós interconectados, capazes de processar informações e aprender padrões a partir de exemplos. Redes neurais são amplamente utilizadas em tarefas como classificação, reconhecimento de imagens e processamento de linguagem natural.

\subsection{Aprendizado Profundo}

Aprendizado Profundo (\emph{Deep Learning}) é uma área do aprendizado de máquina que utiliza redes neurais com múltiplas camadas para aprender representações cada vez mais abstratas dos dados. Conforme o número de camadas aumenta, essas redes se tornam capazes de extrair padrões complexos e hierárquicos, permitindo avanços significativos em diversas tarefas cognitivas tradicionalmente difíceis para sistemas computacionais.

O avanço do aprendizado profundo está diretamente relacionado à disponibilidade de grandes volumes de dados, ao aumento da capacidade computacional e ao desenvolvimento de novas técnicas. Graças ao aprendizado profundo, problemas antes considerados inviáveis passaram a ter soluções com desempenho igual ou superior ao humano em diversos contextos.

\subsection{IA Generativa}


A IA generativa é um ramo da inteligência artificial focado na criação de novos conteúdos, como textos, imagens, códigos, áudios ou vídeos, a partir de padrões aprendidos em grandes conjuntos de dados. Diferentemente de modelos tradicionais, que apenas classificam ou predizem valores específicos, os modelos generativos aprendem distribuições complexas e conseguem produzir saídas originais e coerentes com o contexto. Essa capacidade permitiu o desenvolvimento de aplicações como sistemas de criação de imagens, ferramentas de escrita automatizada, assistentes virtuais avançados e modelos de geração de código.

\subsection{Modelos de Linguagem de Grande Escala (LLMs)}
Entre as principais tecnologias associadas à IA generativa estão os Modelos de Linguagem de Grande Escala (\emph{Large Language Models}, ou LLMs). Esses modelos utilizam a arquitetura de transformadores (\emph{transformers}), que se baseia em mecanismos de atenção (attention mechanisms) para capturar relações de longo alcance entre elementos de uma sequência. Treinados com bilhões de palavras, códigos e documentos, os LLMs podem realizar uma grande variedade de tarefas, incluindo resumo, tradução, classificação, análise semântica, escrita de textos e geração de código.
