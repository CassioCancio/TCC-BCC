%!TeX root=../tese.tex
%("dica" para o editor de texto: este arquivo é parte de um documento maior)
% para saber mais: https://tex.stackexchange.com/q/78101

%% ------------------------------------------------------------------------- %%

% "\chapter" cria um capítulo com número e o coloca no sumário; "\chapter*"
% cria um capítulo sem número e não o coloca no sumário. A introdução não
% deve ser numerada, mas deve aparecer no sumário. Por conta disso, este
% modelo define o comando "\chapter**".
\chapter**{Introdução}\label{cap:introducao}

\section**{Contexto}

A engenharia de \textit{software} é um campo da computação que surgiu e se desenvolveu através da crescente demanda da sociedade do fim do século XX até a atualidade por sistemas computacionais cada vez mais complexos. Neste contexto, ferramentas, métodos e processos foram criados para possibilitar o atendimento dessa demanda.

Nas últimas décadas, os estudos em inteligência artificial (IA) avançaram rapidamente, de modo que um novo paradigma em IA surgiu, a IA generativa. Diferentemente da IA tradicional, a IA generativa tem a capacidade de criar conteúdos novos e originais baseados no que aprendeu, em vez de apenas copiar, imitar e reproduzir algo que já existe.

Dada a flexibilidade e a abertura de diversas possibilidades com essa nova tecnologia, é natural que uma de suas aplicações fosse a engenharia de \textit{software}. Nos últimos anos, diversos estudos foram publicados a fim de analisar essas aplicações, suas consequências e propor diferentes abordagens para tais aplicações. (~\cite{10705752} e~\cite{10.1145/3715003})

Desta forma, este projeto se propõe a levantar os resultados observados em diversos artigos que tratam do estudo do impacto da IA generativa na engenharia de \textit{software}. Além de realizar um estudo de caso sobre a aplicação da IA generativa ao longo das fases do desenvolvimento de \textit{software}, avaliando sua utilidade, limitações e impacto na qualidade e produtividade.

Como estudo de caso, foi realizado o desenvolvimento de uma aplicação web na área de investimentos, com banco de dados, \textit{backend} em \textit{Java   Spring Boot} e \textit{frontend} em \textit{Angular}, utilizando ferramentas de IA generativa nas diferentes fases do desenvolvimento de um sistema. As fases analisadas foram: coleta e análise de requisitos, estudo de viabilidade, \textit{design} de \textit{software}, codificação e testes.

\section**{Motivação}

Este trabalho se faz relevante no contexto em que o uso de ferramentas de IA generativa vem crescendo com o passar dos anos, desde o surgimento de ferramentas como ChatGPT e GitHub Copilot. Segundo dados da \textit{Stack Overflow 2024 Developer Survey}~\citep{StackOverflow2024}, 63,2\% dos desenvolvedores profissionais já utilizam ferramentas de IA no seu processo de desenvolvimento, enquanto que 13.5\% desse mesmo grupo planeja utilizá-las em breve. Além disso, entre os desenvolvedores que responderam usar inteligência artificial, 82\% a utiliza para escrever código.

Desta maneira, é evidente que uma nova tecnologia com amplo uso no mercado de \textit{software} e que abre possibilidade para diversas aplicações, terá impactos sobre como os desenvolvedores escrevem seus códigos. Assim, é de suma importância buscar avaliar e compreender melhor de que maneira esses impactos vêm ocorrendo nas bases de código, inclusive através de um estudo de caso.

\section**{Objetivos}

Os principais objetivos do trabalho são:

\begin{itemize}
  \item Compreender de que maneira a IA generativa tem impactado na produção de código das empresas de \textit{software}, através de um levantamento de dados disponíveis em outros artigos;
  \item Desenvolver um sistema com todo seu processo voltado ao uso de ferramentas de IA generativa ao longo das suas diferentes fases;
  \item Documentar os resultados gerados pela IA durante o processo de desenvolvimento do sistema, incluindo os prompts utilizados;
  \item Analisar os resultados obtidos, a fim de mensurar a qualidade das respostas geradas.
\end{itemize}