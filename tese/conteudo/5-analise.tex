%!TeX root=../tese.tex

\chapter{Análise comparativa}
\label{cap:analise}



	•	produtividade e qualidade de código,
	•	mudança no fluxo de trabalho do desenvolvedor,
	•	confiança e percepção de precisão das ferramentas,
	•	riscos de segurança e privacidade,
	•	impactos educacionais e profissionais,
	•	papel das ferramentas de geração de código.



3.4. Análise da evolução da percepção sobre IA

Um dos objetivos centrais do trabalho é traçar o amadurecimento da percepção sobre IA generativa no desenvolvimento de software. Para isso, são avaliadas mudanças no discurso, nas métricas e nas narrativas observadas nos relatórios ao longo dos últimos anos.

São analisados:
	•	variações de adoção entre anos consecutivos,
	•	mudanças no grau de confiança dos desenvolvedores,
	•	expectativas declaradas para o futuro,
	•	evolução de preocupações (por exemplo: de produtividade → segurança → governança),
	•	transformações no perfil de usuários e suas motivações.

Diferença entre os usos que os desenvolvedores dizem que fariam com IA e o que fazem de fato relacionando isso com a satisfação (já que novas funcionalidades são a coisa preferida);

Analisar resultados que mostram que programadores se sente produtivos aceitando sugestões de ia com gartner e ia agêntica e com gartner e os erros de código não revisado. 

Relacionar os diferentes indicativos de que a adoção da AI continua, gartners e overflow, dando robustez à análise e sugerindo a continuidade da tendência. Comentar também redução expressiva do tráfego do Stack Overflow que apareceu inclusive na forte diminuição de respondentes da developer survey.

Evidenciar o problema no longo prazo, que isso é só o começo da tendência.
