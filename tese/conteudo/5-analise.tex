%!TeX root=../tese.tex

\chapter{Análise}
\label{cap:analise}

Neste capítulo, com base em todos os dados levantados, foi realizada uma análise que reuniu os diferentes trabalhos e cruzou as informações mais importantes entre si. Os artigos e relatórios foram escolhidos porque abordam diferentes aspectos do tema, desde um ponto de vista da indústria até abordagens acadêmicas.

\section{Amadurecimento da percepção dos programadores}

A mudança no sentimento dos desenvolvedores a respeito da IA ficou bastante visível na compilação dos resultados das pesquisas do \emph{Stack Overflow} (\cite*{StackOverflow2023,StackOverflow2024,StackOverflow2025}). Apesar de o uso de ferramentas de IA continuar crescendo (\autoref{fig:historico_uso_ia}), tanto a percepção dos desenvolvedores sobre a qualidade das respostas das ferramentas de IA para a engenharia de \emph{software} (\autoref{fig:percepcao_ia}) quanto a confiança dos profissionais em seus conteúdos gerados (\autoref{fig:confianca_ia}) pioraram significativamente nos últimos anos. 

Isso talvez explique, o aumento na proporção de tarefas as quais os desenvolvedores se apresentam desinteressados a adotar (\autoref{tab:nao_interesse_ia}). Ano após ano, os programadores apresentaram mais resistência a usar IA para atividades de alta responsabilidade ou de complexidade sistêmica como \emph{deploy} e monitoramento ou planejamento de projetos (\autoref{tab:nao_interesse_ia}).

\section{Indícios da adoção e extrapolação}

Este trabalho reuniu diversos relatórios que estimam o uso de ferramentas de IA por parte dos desenvolvedores. Dado o tamanho das amostras destas pesquisas, não é possível afirmar com relevância estatística que de fato todos os programadores do mundo estão adotando estas ferramentas de IA nestas proporções, no entanto, há alguns indícios interessantes. 

A \emph{Stack Overflow} (\cite*{StackOverflow2023,StackOverflow2024,StackOverflow2025}) tem apresentado consistentemente um aumento nas declarações de programadores que utilizam IA (\autoref{fig:historico_uso_ia}). Outra evidência, também relacionada ao \emph{Stack Overflow}, foi documentada por \citet{del_Rio_Chanona_2024}, que analisou a forte queda nos acessos à plataforma, relacionando-a com o aumento do uso de ferramentas de IA para tirar dúvidas de código ou fazer triagem de erros. 

Além disso, a própria Gartner, através do relatório escrito por \citet{gartner2024agenticAI} e do escrito por \citet{gartner_ai_code_assistants_2024}, evidencia suas previsões de aumento tanto do uso de IA generativa em assistentes de código quanto através de ferramentas de IA agêntica. A pesquisa de \citet{elsevier} também corrobora com essa visão de aumento, já que 84,2\% dos participantes da pesquisa declarou utilizar assistentes de IA.

\section{Desconexão entre discurso e prática}

Como evidenciado no artigo de \citet{elsevier}, os programadores entrevistados se declararam mais propensos a delegar tarefas que gostam menos para as ferramentas de IA (\autoref{fig:grafico_gosto_delegacao}), no entanto, a própria pesquisa evidencia que o maior uso das ferramentas de IA entre os participantes é justamente a produção de código, a tarefa apontado na pesquisa como a mais proveitosa pelos programadores e a que eles menos delegariam a uma IA (\autoref{fig:grafico_gosto_delegacao}). Este é um ponto interessante a se avaliar em pesquisas futuras, no sentido do impacto dessa ação para a satisfação profissional da categoria.

\section{Tendência de automatização}

Como evidenciado pelo artigo de \citet{gartner2024agenticAI}, da \emph{Gartner}, o mercado não apenas caminha para a continuidade do uso de ferramentas de IA nos diversos setores empresariais, incluindo a produção de \emph{software}, mas também para o crescimento da adoção da IA agêntica. No contexto de empresas de \emph{software}, a IA agêntica não apenas sugere e gera código, como também manipula a base de código diretamente. 

Segundo \citet{10.1145/3520312.3534864}, os programadores tendem a se sentir mais produtivos conforme aceitam mais sugestões de seus assistentes de código. Essa informação somada à previsão da \emph{Gartner} apresentada por \citet{gartner_ai_code_assistants_2024} de que 25\% dos erros de produção serão advindos de código de IA não revisado, já indicam a necessidade da criação de mecanismo que impeçam os programadores de adicionar linhas não revisadas às bases de código. Isto será especialmente desafiador dado o crescimento esperado da IA agêntica.


\section{As métricas e o futuro}

A queda na qualidade das métricas evidenciado pela \citet{gitclear2025} apresenta não apenas um cenário negativo em relação ao aumento da duplicação de código, quanto da queda da refatoração. Neste contexto, é relevante investigar como essa piora da qualidade impactará a sustentabilidade das empresas e de que maneira essas métricas evoluirão, conforme a adoção de IA avançar.
