%!TeX root=../tese.tex

\chapter{Estratégia e seleção das fontes}%
\label{cap:selecao_estrategia}

Este capítulo descreve a abordagem metodológica adotada para a condução da pesquisa, detalhando os critérios utilizados para a seleção das fontes, a organização das análises e o método empregado para a comparação entre literatura formal e literatura cinzenta. O trabalho segue um processo analítico que parte de uma fonte central de literatura formal, a partir da qual outras fontes, de literatura cinzenta, foram selecionadas a fim de complementar, contrastar e aprofundar a análise.

\section{Seleção das fontes}

A seleção das fontes teve como objetivo garantir relevância temática, atualidade e diversidade de perspectivas. Dessa forma, foram consideradas tanto fontes acadêmicas quanto fontes oriundas do mercado, incluindo estudos empíricos, relatórios técnicos, pesquisas de opinião e análises baseadas em métricas de código. Essa combinação permitiu contemplar diferentes pontos de vista, como os de pesquisadores, desenvolvedores e executivos, bem como evidências quantitativas extraídas de bases de código reais.

Na engenharia de \emph{software}, diferentemente de muitas outras áreas do conhecimento, conferências e congressos frequentemente possuem peso equivalente ou até superior ao de periódicos científicos tradicionais, em função da rápida evolução tecnológica da área. Por esse motivo, a seleção das fontes não se restringiu a artigos publicados em periódicos acadêmicos, mas incluiu também relatórios técnicos e pesquisas de mercado amplamente utilizados na indústria. As fontes foram organizadas em dois grupos: literatura formal e literatura cinzenta.

\subsection{Literatura formal}

No grupo de literatura formal, o artigo de \citet{elsevier} foi adotado como ponto de partida da pesquisa. Essa escolha se justifica por se tratar da revisão de literatura mais recente identificada no momento da seleção das fontes, publicada no primeiro semestre de 2025, além de apresentar ampla cobertura dos aspectos técnicos e práticos relacionados ao uso de assistentes de código baseados em inteligência artificial.

O artigo encontra-se indexado em bases reconhecidas, como o \emph{Information and Software Technology} da \emph{Elsevier}, um periódico relevante na área de engenharia de \emph{software}. Trata-se de um estudo revisado por pares, conduzido por autores com forte vínculo com a indústria, majoritariamente pesquisadores da \emph{JetBrains}, o que favorece uma abordagem alinhada a práticas reais de desenvolvimento. O artigo investiga como desenvolvedores utilizam assistentes de inteligência artificial, as razões para sua não adoção em determinadas etapas do SDLC e os aspectos que necessitam de aprimoramento, tornando-o uma referência adequada para estruturar e orientar a análise realizada neste trabalho.

\subsection{Literatura cinzenta}

O grupo de literatura cinzenta foi composto por relatórios e estudos que pudessem complementar a análise sob diferentes perspectivas empíricas e de mercado. O relatório de \citet{gitclear2025} concentra-se na análise de métricas de código extraídas de repositórios abertos e de empresas, construindo uma série histórica de resultados entre 2020 e 2024. A relevância dessa fonte decorre do fato de a GitClear atuar diretamente na análise automatizada de repositórios de código em larga escala, possuindo acesso contínuo a bases de código reais tanto de projetos de código aberto quanto de ambientes corporativos.

As pesquisas conduzidas pelo \citeauthor{StackOverflow2023} \parencite*{StackOverflow2023,StackOverflow2024,StackOverflow2025} compilam tendências relacionadas ao perfil e aos hábitos de uso de ferramentas de IA por desenvolvedores usuários da plataforma. A relevância dessas fontes está associada à posição do \emph{Stack Overflow} como uma das maiores comunidades globais de desenvolvedores, reunindo milhões de usuários ativos de diferentes níveis de experiência, setores e regiões. Assim, os dados coletados refletem percepções e práticas amplamente disseminadas na comunidade de desenvolvimento de \emph{software}. Neste trabalho, essas pesquisas foram organizadas de forma comparativa, possibilitando a análise de mudanças temporais e a identificação de padrões recorrentes no uso de ferramentas de IA.

Os relatórios de \citet{gartner2024agenticAI} e \citet{gartner_ai_code_assistants_2024}, publicados pela \emph{Gartner Research}, oferecem uma perspectiva de mercado sobre o uso de inteligência artificial generativa no desenvolvimento de \emph{software}, com foco em tendências emergentes, riscos percebidos e expectativas de executivos e gestores de tecnologia. A \emph{Gartner} é reconhecida internacionalmente como uma das principais consultorias de pesquisa e análise estratégica em tecnologia da informação, sendo amplamente utilizada por organizações para embasar decisões de investimento e adoção tecnológica. Por esse motivo, seus relatórios fornecem uma visão consolidada das expectativas e preocupações do ponto de vista organizacional e gerencial.

Por fim, o relatório de \citet{anssi_2024} foi produzido em colaboração por duas agências governamentais europeias, a francesa, ANSSI, e a alemã, BSI, ambas responsáveis por diretrizes e políticas de segurança da informação em seus respectivos países. O documento reúne e sintetiza resultados de pesquisas acadêmicas relacionadas ao uso de assistentes de código baseados em inteligência artificial, com ênfase em riscos, limitações e implicações de segurança. Essa fonte ocupa uma posição intermediária entre literatura formal e cinzenta, sendo utilizada neste trabalho tanto como uma consolidação de evidências acadêmicas quanto como um arcabouço de referência para embasar e estruturar a análise comparativa.

\section{Análises}

O \autoref{cap:fontes} foi dedicado à apresentação detalhada das fontes de literatura formal e cinzenta analisadas neste trabalho. Para cada fonte, investigou-se o contexto da organização responsável, o método empregado para a coleta e análise dos dados e os resultados mais relevantes para os objetivos da pesquisa. A \autoref{sec:literatura-formal} concentra-se na apresentação da fonte de literatura formal, enquanto a \autoref{sec:literatura-cinzenta} aborda as fontes de literatura cinzenta.

Os conteúdos apresentados no \autoref{cap:fontes} serviram de base para a análise desenvolvida no \autoref{cap:analise}, cujo objetivo foi identificar consensos e divergências entre as diferentes fontes. Além disso, nesse capítulo, o relatório de \citet{anssi_2024} é utilizado de forma mais direta como um arcabouço de referência acadêmica, auxiliando na contextualização e no embasamento dos resultados discutidos.