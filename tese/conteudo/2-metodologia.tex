%!TeX root=../tese.tex
%("dica" para o editor de texto: este arquivo é parte de um documento maior)
% para saber mais: https://tex.stackexchange.com/q/78101

% Vamos definir alguns comandos auxiliares para facilitar.

% "textbackslash" é muito comprido.
\newcommand{\sla}{\textbackslash}

% Vamos escrever comandos (como "make" ou "itemize") com formatação especial.
\newcommand{\cmd}[1]{\textsf{#1}}

% Idem para packages; aqui estamos usando a mesma formatação de \cmd,
% mas poderíamos escolher outra.
\newcommand{\pkg}[1]{\textsf{#1}}

% A maioria dos comandos LaTeX começa com "\"; vamos criar um
% comando que já coloca essa barra e formata com "\cmd".
\newcommand{\ltxcmd}[1]{\cmd{\sla{}#1}}

\chapter{Metodologia}%
\label{cap:metodologia}

Este capítulo descreve a abordagem metodológica adotada para condução da pesquisa, destacando os critérios utilizados para seleção das fontes, a forma de organização das análises e o método empregado para comparar literatura cinzenta e literatura formal. O trabalho segue um processo de análise partindo do artigo de literatura formal escrito por \citet{elsevier}. A partir deste artigo, outras fontes, principalmente de literatura cinzenta, foram selecionadas de modo a complementar a análise. 

\section{Seleção das fontes}

A seleção das fontes foi feita buscando garantir a relevância, atualidade e diversidade de perspectivas. Desta forma, as fontes incluem perspectivas acadêmicas e de mercado, impressões de programadores e executivos e análises de métricas de código. As fontes foram divididas no grupo de literatura formal e literatura cinzenta.

\subsection{Literatura formal}

Neste grupo, o artigo de \citet{elsevier}, indexado em bases reconhecidas, como a \emph{Elsevier}, foi utilizado como ponto de partida do trabalho. Também parte deste grupo, o artigo \citet{anssi_2024} foi produzido por duas agências governamentais europeias, a francesa, ANSSI, e a alemã, BSI, e é um compilado de resultados de pesquisas acadêmicas referentes ao uso de assistentes de código com IA.

\subsection{Literatura cinzenta}

Neste grupo, foram trazidos relatórios que pudessem complementar a análise. O artigo da \citet{gitclear2025} foca na análise de métricas de código em repositórios abertos e de empresas, criando um histórico dos resultados de 2020 a 2024. As pesquisas do \citeauthor{StackOverflow2023} \parencite*{StackOverflow2023,StackOverflow2024,StackOverflow2025} compilam tendências no perfil de uso de IA dos usuários da plataforma e, neste trabalho, foram organizados de modo a criar um comparativo entre os anos. Os artigos de \citet{gartner2024agenticAI} e \citet{gartner_ai_code_assistants_2024} da \emph{Gartner Research} trazem uma visão de mercado sobre o uso de IA generativa no desenvolvimento de software, focando em tendências, riscos e expectativas de executivos.

% TODO - garantir que esse capítulo continua correto)
\section{Análises}
No capítulo \autoref{cap:literatura-formal}, o foco é apresentar o texto de \citet{elsevier}, o ponto de partida do trabalho. O artigo aborda principais ferramentas de IA utilizadas pelos participantes da pesquisa, suas tendências de uso e impressões sobre o desempenho das ferramentas.

O capítulo \autoref{cap:literatura-cinzenta} é dedicado às fontes de literatura cinzenta. Nele estão apresentados um resumo dos trabalhos, seus resultados relevantes para este trabalho e um breve contexto da organização por trás de cada fonte. 

Por fim, os conteúdos dos dois capítulos foram utilizados na análise realizada no \autoref{cap:analise}, cujo objetivo foi evidenciar consensos e divergências entre as fontes. Além disso, neste capítulo, o artigo \citet{anssi_2024} foi utilizado como arcabouço de fontes acadêmicas, ajudando a complementar e embasar a análise.
