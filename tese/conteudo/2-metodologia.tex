%!TeX root=../tese.tex
%("dica" para o editor de texto: este arquivo é parte de um documento maior)
% para saber mais: https://tex.stackexchange.com/q/78101

% Vamos definir alguns comandos auxiliares para facilitar.

% "textbackslash" é muito comprido.
\newcommand{\sla}{\textbackslash}

% Vamos escrever comandos (como "make" ou "itemize") com formatação especial.
\newcommand{\cmd}[1]{\textsf{#1}}

% Idem para packages; aqui estamos usando a mesma formatação de \cmd,
% mas poderíamos escolher outra.
\newcommand{\pkg}[1]{\textsf{#1}}

% A maioria dos comandos LaTeX começa com "\"; vamos criar um
% comando que já coloca essa barra e formata com "\cmd".
\newcommand{\ltxcmd}[1]{\cmd{\sla{}#1}}

\chapter{Metodologia}%
\label{cap:metodologia}

\section{Abordagem de Pesquisa}
\subsection{Tipo de Pesquisa}
Esta pesquisa caracteriza-se como descritiva, dado que se fundamenta na análise de produções teóricas previamente publicadas sobre o tema. Para isso, foram utilizados livros, artigos científicos, relatórios técnicos e trabalhos acadêmicos que abordem o impacto da inteligência artificial generativa na engenharia de software, bem como suas aplicações nas fases de design, codificação e testes.

\subsection{Procedimentos Metodológicos}
Quanto à sua finalidade, trata-se de uma pesquisa \textbf{básica}, pois tem como objetivo aprofundar o conhecimento científico existente sobre o tema, explorando especificamente os efeitos e implicações do uso de ferramentas de IA generativa no desenvolvimento de software. Dessa forma, não se busca propor soluções práticas imediatas, mas compreender e organizar criticamente o conhecimento já produzido.

\section{Coleta de Dados}
\subsection{Fontes de Dados}
A seleção do material bibliográfico será feita em bases de dados científicas, como ACM Digital Library, IEEE Xplore, Scopus e Google Scholar, utilizando termos relacionados à inteligência artificial generativa, engenharia de software, ferramentas de apoio ao desenvolvimento e impactos na prática profissional.

\section{Métodos de Análise}
No que se refere à abordagem, este trabalho adota uma pesquisa qualitativa. Os estudos selecionados serão analisados criticamente, considerando seus objetivos, métodos, resultados e conclusões. A interpretação dos dados terá caráter descritivo e analítico, buscando identificar padrões, benefícios, desafios e lacunas nas aplicações da IA generativa no contexto da engenharia de software.