%!TeX root=../tese.tex
%("dica" para o editor de texto: este arquivo é parte de um documento maior)
% para saber mais: https://tex.stackexchange.com/q/78101

% Vamos definir alguns comandos auxiliares para facilitar.

% "textbackslash" é muito comprido.
\newcommand{\sla}{\textbackslash}

% Vamos escrever comandos (como "make" ou "itemize") com formatação especial.
\newcommand{\cmd}[1]{\textsf{#1}}

% Idem para packages; aqui estamos usando a mesma formatação de \cmd,
% mas poderíamos escolher outra.
\newcommand{\pkg}[1]{\textsf{#1}}

% A maioria dos comandos LaTeX começa com "\"; vamos criar um
% comando que já coloca essa barra e formata com "\cmd".
\newcommand{\ltxcmd}[1]{\cmd{\sla{}#1}}

\chapter{Metodologia}%
\label{cap:metodologia}

Este capítulo descreve a abordagem metodológica adotada para condução da pesquisa, destacando os critérios utilizados para seleção das fontes, a forma de organização das análises e o método empregado para comparar literatura cinzenta e literatura formal. O trabalho segue um processo de análise partindo do artigo de literatura formal escrito por \citet{elsevier}. A partir deste artigo, outras fontes, de literatura cinzenta, foram selecionadas de modo a complementar a análise, oferecendo outras visões sobre o uso de ferramentas de IA generativa no desenvolvimento de software e seus impactos.

As fontes utilizadas neste trabalho foram divididas em dois grupos:

\begin{itemize}
    \item \textbf{Literatura formal:} neste grupo há o artigo científico principal criado por \citet{elsevier}, que está indexado em bases reconhecidas, como a \emph{Elsevier}, e foi utilizado como ponto de partida da análise. Além dele, há o relatório, \citet{anssi_2024}, produzido por duas agências governamentais, a francesa, ANSSI, e a alemã, BSI, que compila diversos resultados de pesquisas acadêmicas referentes ao uso de assistentes de código com IA;
    \item \textbf{Literatura cinzenta:} composta por quatro relatórios internacionais de organizações influentes na indústria de software e tecnologia. Esses reports foram escolhidos porque:
    \begin{itemize}
        \item apresentam dados atualizados sobre o uso de IA por desenvolvedores;
        \item refletem tendências observadas no mercado;
        \item oferecem recortes e metodologias distintas, permitindo observar convergências e discrepâncias;
        \item possuem grande alcance e impacto na comunidade técnica.
    \end{itemize}
\end{itemize}
    
A análise das fontes foi estruturada em três etapas principais:
(1) Panorama inicial dos reports

Cada uma das fontes é descrita no Capítulo 2, quanto ao seu conteúdo, escopo, metodologia empregada e perfil da organização responsável, permitindo compreender seus vieses e áreas de foco. 

Primeiro, é apresentado um panorama geral dos quatro relatórios selecionados, destacando:
	•	seus objetivos declarados,
	•	tipos de dados coletados,
	•	público-alvo,
	•	principais temas abordados.

Essa etapa permite estabelecer uma ordem de foco, evidenciando que cada documento enfatiza diferentes dimensões do uso de IA, como produtividade, riscos, experiência de desenvolvedores ou impactos organizacionais.

(2) Análise dos assuntos em foco

Com base nos temas recorrentes entre os reports e no artigo principal, foram identificados assuntos centrais para investigação, tais como:
	•	produtividade e qualidade de código,
	•	mudança no fluxo de trabalho do desenvolvedor,
	•	confiança e percepção de precisão das ferramentas,
	•	riscos de segurança e privacidade,
	•	impactos educacionais e profissionais,
	•	papel das ferramentas de geração de código.

A análise procura destacar variações entre os resultados, evidenciando tanto consensos quanto divergências entre literatura formal e cinzenta.

(3) Comparação entre literatura cinzenta e literatura formal

Por fim, os achados dos reports da indústria são comparados com o artigo científico da Elsevier, buscando identificar:
	•	confirmações (quando ambos os grupos de fontes apontam para as mesmas tendências);
	•	gaps (temas presentes na literatura cinzenta, mas ainda pouco explorados academicamente, ou vice-versa);
	•	contradições (resultados divergentes entre academia e indústria).

Essa comparação permite avaliar em que medida a prática acompanha, complementa ou diverge da teoria apresentada na literatura formal.





3.4. Análise da evolução da percepção sobre IA

Um dos objetivos centrais do trabalho é traçar o amadurecimento da percepção sobre IA generativa no desenvolvimento de software. Para isso, são avaliadas mudanças no discurso, nas métricas e nas narrativas observadas nos relatórios ao longo dos últimos anos.

São analisados:
	•	variações de adoção entre anos consecutivos,
	•	mudanças no grau de confiança dos desenvolvedores,
	•	expectativas declaradas para o futuro,
	•	evolução de preocupações (por exemplo: de produtividade → segurança → governança),
	•	transformações no perfil de usuários e suas motivações.

Essa análise longitudinal complementa o estudo comparativo entre fontes e permite compreender como a prática tem se transformado em resposta ao avanço das ferramentas generativas.