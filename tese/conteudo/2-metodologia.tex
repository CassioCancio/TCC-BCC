%!TeX root=../tese.tex

\chapter{Estratégia e seleção das fontes}%
\label{cap:selecao_estrategia}

Este capítulo descreve a abordagem metodológica adotada para a condução da pesquisa, destacando os critérios utilizados para seleção das fontes, a forma de organização das análises e o método empregado para comparar literatura cinzenta e literatura formal. O trabalho segue um processo de análise partindo do artigo de literatura formal escrito por \citet{elsevier}. A partir deste artigo outras fontes, de literatura cinzenta, foram selecionadas de modo a complementar a análise. 

\section{Seleção das fontes}

A seleção das fontes foi feita buscando garantir a relevância, atualidade e diversidade de perspectivas. Desta forma, as fontes incluem perspectivas acadêmicas e de mercado, impressões de programadores e executivos e análises de métricas de código. 

Na engenharia de \emph{software}, diferentemente de muitas outras ciências, as conferências e congressos podem ter um peso quase igual ou superior aos jornais e revistas, devido à velocidade de evolução da área. Por isso, a seleção das fontes não se restringiu apenas a artigos de periódicos acadêmicos, mas também incluiu relatórios técnicos e pesquisas de mercado. 
Elas foram divididas nos grupos de literatura formal e literatura cinzenta.

% TODO - ajustar metodologia após junção dos capítulos de literatura cinzenta e formal
\subsection{Literatura formal}

Neste grupo, o artigo de \citet{elsevier}, indexado em bases reconhecidas, como a \emph{Elsevier}, foi utilizado como ponto de partida do trabalho. 
% TODO - melhorar justificativa
% -- Precisamos justificar melhor as escolhas/seleção. No caso, deixar claro que é a mais recente revisão de literatura que encontrarmos no momento da seleção dos trabalhos no primeiro semestre de 2025, com uma boa abrangência/cobertura de aspectos técnicos, publicada em um journal relevante na ES, com alto fator de impacto, e os autores são da indústria (jetbrains).
% -- Fazer o mesmo e justificar melhor as demais fontes; vi que, mais à frente, você apresenta melhor, mas precisamos justificá-las melhor aqui.


\subsection{Literatura cinzenta}

Neste grupo, foram reunidos relatórios que pudessem complementar a análise. O artigo da \citet{gitclear2025} foca na análise de métricas de código em repositórios abertos e de empresas, criando um histórico dos resultados de 2020 a 2024. As pesquisas do \citeauthor{StackOverflow2023} \parencite*{StackOverflow2023,StackOverflow2024,StackOverflow2025} compilam tendências no perfil de uso de IA dos usuários da plataforma e, neste trabalho, foram organizadas de modo a criar um comparativo entre os anos. Os artigos de \citet{gartner2024agenticAI} e \citet{gartner_ai_code_assistants_2024} da \emph{Gartner Research} trazem uma visão de mercado sobre o uso de IA generativa no desenvolvimento de \emph{software}, focando em tendências, riscos e expectativas de executivos. Também parte deste grupo, o artigo \citet{anssi_2024} foi produzido por duas agências governamentais europeias, a francesa, ANSSI, e a alemã, BSI, e é um compilado de resultados de pesquisas acadêmicas referentes ao uso de assistentes de código com IA.
% TODO - melhorar justificativas de cada artigo

% TODO - corrigir
\section{Análises}
% No \autoref{cap:literatura-formal}, o foco é apresentar o texto de \citet{elsevier}, o ponto de partida do trabalho. O artigo aborda principais ferramentas de IA utilizadas pelos participantes da pesquisa, suas tendências de uso e impressões sobre o desempenho das ferramentas. Além disso, um breve contexto sobre o \citet{anssi_2024} é apresentado sem aprofundar tanto no conteúdo, já que ele é bastante denso.

Já o \autoref{cap:literatura-cinzenta} dedica-se às fontes de literatura cinzenta. Nele estão apresentados um resumo dos relatórios, seus resultados relevantes para este trabalho e um breve contexto da organização por trás de cada fonte. 

Por fim, os conteúdos dos dois capítulos foram utilizados na análise realizada no \autoref{cap:analise}, cujo objetivo foi evidenciar consensos e divergências entre as fontes. Além disso, neste capítulo, o artigo \citet{anssi_2024} é utilizado de forma mais completa como arcabouço de fontes acadêmicas, ajudando a complementar e embasar a análise.
