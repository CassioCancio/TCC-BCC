%!TeX root=../tese.tex

\chapter{Literatura formal}
\label{cap:literatura-formal}

\section{\citetitle{elsevier} \parencite*{elsevier}}

\section{\citetitle{anssi_2024} \parencite*{anssi_2024}}

O relatório foi elaborado conjuntamente pela \emph{Agence Nationale de la Sécurité des Systèmes d’Information} (ANSSI), agência francesa responsável pela cibersegurança nacional, e pelo \emph{Bundesamt für Sicherheit in der Informationstechnik} (BSI), o Escritório Federal Alemão para Segurança da Informação. Ambas são autoridades governamentais que atuam na definição de diretrizes, recomendações e boas práticas para a proteção de sistemas de informação e infraestrutura digital em seus respectivos países.

A publicação analisa as oportunidades e os riscos associados ao uso de assistentes de programação baseados em IA, tecnologia que já é amplamente adotada em organizações e tende a se tornar parte integrante do desenvolvimento de software. O relatório começa com uma introdução conceitual sobre o que são assistentes de código com IA e o objetivo do documento. Em seguida, o texto é dividido em seções temáticas que discutem, separadamente, as oportunidades associadas ao uso dessas ferramentas, os riscos de segurança envolvidos e, por fim, um conjunto de recomendações práticas. A conclusão consolida essas recomendações, segmentando-as por níveis organizacionais (gestão, desenvolvimento e pesquisa), o que confere ao texto um caráter normativo e orientado à aplicação prática, típico de publicações institucionais de órgãos de segurança da informação.

O conteúdo em detalhes do artigo não foi compilado na íntegra neste trabalho, pois é bastante denso e repleto de citações diversas. Em vez disso, seu conteúdo e suas referências serão utilizados diretamente na análise presente no \autoref{cap:analise}. 