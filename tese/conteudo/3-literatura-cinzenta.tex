%!TeX root=../tese.tex

\chapter{Literatura Cinzenta}
\label{cap:literatura-cinzenta}

A literatura cinzenta é o nome dado para publicações, textos e produções os quais não passaram pelo mesmo processo de revisão por pares que uma publicação comercial, por isso, é importante verificá-la. Neste contextos, foi realizada uma seleção de artigos e publicações nesta categoria de modo a comparar de que maneira esses relatórios se relacionam com a literatura branca.

Este capítulo se propõe a apresentar os textos de literatura utilizados ao longo do trabalho

\section{Texto1}


% Teses e dissertações, anais de conferências, boletins informativos, relatórios, documentos governamentais e parlamentares, comunicações informais, traduções, dados de censo, relatórios de pesquisa, relatórios técnicos, padrões, patentes, vídeos, ensaios clínicos e diretrizes práticas, eprints, preprints, artigos wiki, e-mails, blogs, arquivos de dados de pesquisa e dados científicos, levantamentos geológicos e geofísicos, mapas, conteúdo de repositórios.

