%!TeX root=../tese.tex
%("dica" para o editor de texto: este arquivo é parte de um documento maior)
% para saber mais: https://tex.stackexchange.com/q/78101

% As palavras-chave são obrigatórias, em português e em inglês, e devem ser
% definidas antes do resumo/abstract. Acrescente quantas forem necessárias.
\palavraschave{IA Generativa, Engenharia de Software, Assistente de Código de IA}


\keywords{Generative AI, Software Engineering, AI Code Assistant}

% O resumo é obrigatório, em português e inglês. Estes comandos também
% geram automaticamente a referência para o próprio documento, conforme
% as normas sugeridas da USP.
\resumo{
Este trabalho visa reunir e analisar dados sobre os impactos da IA generativa na produção de software, focando na evolução da percepção dos programadores e comparando diferentes pesquisas e relatórios sobre o tema. A metodologia adotada envolveu a seleção e a análise de relatórios de mercado de 2023 a 2025 e de revisões recentes da literatura. Os resultados demonstram um crescimento significativo na adoção de ferramentas de IA por desenvolvedores profissionais, atingindo 80,7\%, em uma das pesquisas analisadas. No entanto, essa adoção foi acompanhada por uma queda na percepção de qualidade e no nível de confiança dos profissionais sobre os resultados gerados pela IA. Paralelamente, uma análise de métricas de código, como o aumento do churn e da duplicação, e a queda da refatoração, sugere uma deterioração da qualidade do código. O estudo também aponta uma discrepância entre o uso predominante da IA para escrita de código, considerada a atividade mais agradável pelos desenvolvedores, e o desejo declarado de delegar tarefas menos prazerosas, como a escrita de testes. O cenário atual indica a continuidade da expansão da IA na engenharia de software, especialmente com a ascensão da IA agêntica, o que reforça a necessidade de novos estudos para mitigar os riscos associados à qualidade do código.
}

\abstract{
This work aims to gather and analyze data on the impacts of generative AI on software development, focusing on the evolution of developers' perceptions and comparing different research studies and reports on the subject. The adopted methodology involved the selection and analysis of market reports from 2023 to 2025, as well as recent literature reviews. The results show a significant increase in the adoption of AI tools by professional developers, reaching 80.7\% in one of the analyzed studies. However, this adoption was accompanied by a decline in perceived quality and in the level of confidence professionals place in AI-generated outputs. In parallel, an analysis of code metrics—such as increased churn and duplication, and a decrease in refactoring—suggests a deterioration in code quality. The study also highlights a discrepancy between the predominant use of AI for code writing, considered the most enjoyable activity by developers, and the expressed desire to delegate less pleasurable tasks, such as writing tests. The current scenario indicates the continued expansion of AI in software engineering, especially with the rise of agentic AI, reinforcing the need for further studies to mitigate risks associated with code quality.
}
