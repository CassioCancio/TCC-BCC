%!TeX root=../tese.tex
%("dica" para o editor de texto: este arquivo é parte de um documento maior)
% para saber mais: https://tex.stackexchange.com/q/78101

%% ------------------------------------------------------------------------- %%

% "\chapter" cria um capítulo com número e o coloca no sumário; "\chapter*"
% cria um capítulo sem número e não o coloca no sumário. A introdução não
% deve ser numerada, mas deve aparecer no sumário. Por conta disso, este
% modelo define o comando "\chapter**".
\chapter**{Introdução}\label{cap:introducao}

A engenharia de \emph{software} é um campo da computação que se propõe a produzir e manter sistemas de \emph{software}. Essa definição foi estabelecida em 1968 pela OTAN, direcionando esforços para resolver a chamada ``crise do \emph{software}'', um período em que o desenvolvimento de programas se tornava cada vez mais complexo e desorganizado. Desde então, diversas ferramentas, métodos e processos foram criados para possibilitar que programadores organizassem a produção de \emph{software} e realizassem projetos complexos com maior eficiência.

No mesmo período, a humanidade presenciou diversos avanços tecnológicos, como a produção de processadores cada vez mais potentes, o barateamento do \emph{hardware}, tornando computadores e celulares muito mais acessíveis, e a inclusão de bilhões de pessoas na \emph{internet}, gerando uma enorme quantidade de dados sobre os diversos aspectos da vida cotidiana e virtual. Com todo esse poder computacional e a quantidade massiva de dados disponíveis, a inteligência artificial pôde se desenvolver a passos largos, culminando no surgimento da IA generativa. Diferentemente da IA tradicional, a IA generativa é capaz de ``criar'' conteúdos com base no que aprendeu.

Dada sua flexibilidade, a IA generativa pode ser utilizada para diversos fins, e era natural que uma de suas aplicações fosse a engenharia de \emph{software}. Nos últimos anos, vários estudos foram publicados com o objetivo de analisar essas aplicações, discutir suas consequências e propor abordagens seguras e responsáveis para seu uso (\cite{10705752} e\cite{10.1145/3715003}).

Segundo dados da \citet{StackOverflow2025}, 80,7\% dos desenvolvedores profissionais utilizam ferramentas de IA em seu processo de desenvolvimento de \emph{software} e 4,6\% desse grupo planeja utilizá-las em breve. É evidente que uma tecnologia amplamente adotada entre desenvolvedores tende a causar impactos significativos na produção de código e, nesse contexto, este trabalho faz-se relevante. Seus objetivos são: reunir dados sobre os impactos das ferramentas de IA generativa na produção de \emph{software}, analisar a evolução da percepção dos desenvolvedores sobre o uso dessas ferramentas e comparar os resultados provenientes da literatura cinzenta e da literatura formal.

O \autoref{cap:fundamentacao} apresenta os conceitos fundamentais para o desenvolvimento do trabalho. O \autoref{cap:metodologia} descreve a metodologia adotada e as fontes utilizadas. No \autoref{cap:literatura-cinzenta}, é apresentada uma revisão da literatura cinzenta sobre o tema, já no \autoref{cap:literatura-formal}, a revisão foca na literatura formal. O \autoref{cap:analise} traz uma análise que compara e relaciona os estudos das duas frentes. Por fim, o \autoref{cap:conclusao} conclui o trabalho.