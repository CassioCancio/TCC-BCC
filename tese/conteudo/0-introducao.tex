%!TeX root=../tese.tex
%("dica" para o editor de texto: este arquivo é parte de um documento maior)
% para saber mais: https://tex.stackexchange.com/q/78101

%% ------------------------------------------------------------------------- %%

% "\chapter" cria um capítulo com número e o coloca no sumário; "\chapter*"
% cria um capítulo sem número e não o coloca no sumário. A introdução não
% deve ser numerada, mas deve aparecer no sumário. Por conta disso, este
% modelo define o comando "\chapter**".
\chapter**{Introdução}\label{cap:introducao}

A engenharia de \emph{software} é campo da computação que se propõe a produzir e manter sistemas de \emph{software}. Esta definição foi criada em 1968 pela OTAN, direcionando esforços na resolução da chamada "crise do \emph{software}", um período em que o desenvolvimento de programas se tornava cada vez mais complexo e desorganizado. Desde então, diversas ferramentas, métodos e processos foram criados para possibilitar que programadores pudessem organizar a produção de \emph{software} e realizar projetos complexos.

Neste mesmo período, a humanidade presenciou diversas avanços na tecnologia, como a produção de processadores cada vez mais potentes, o barateamento do \emph{hardware}, tornando computadores e celulares muito mais acessíveis, e a inclusão de bilhões de pessoas na \emph{internet}, gerando uma enorme quantidade de dados sobre os diversos aspectos da vida cotidiana e virtual. Com todo esse poder computacional e dados disponíveis, a inteligência artificial pôde se desenvolver a passos largos, até o surgimento da IA generativa. Diferentemente da IA tradicional, a IA generativa é capaz de "criar" conteúdos baseados no que aprendeu.

Dada a sua flexibilidade, a IA generativa pode ser utilizada para diversos fins e era natural que uma de suas aplicações fosse a engenharia de \emph{software}. Nos últimos anos, diversos estudos foram publicados de modo a analisar essas aplicações, suas consequências e propor diferentes abordagens seguras e responsáveis para tais aplicações (~\cite{10705752} e~\cite{10.1145/3715003}).

Segundo dados da \citet{StackOverflow2025}, 80,7\% dos desenvolvedores profissionais utilizam ferramentas de IA no seu processo de desenvolvimento de \emph{software} e 4.6\% deste grupo planeja utilizá-las em breve. É evidente que uma tecnologia com amplo uso dos desenvolvedores causaria impactos na produção de código e, neste contexto, faz-se relevante este trabalho, que tem como objetivos: reunir dados sobre os impactos das ferramentas de IA generativa na produção de \emph{software}, analisar a evolução da percepção dos desenvolvedores sobre o uso destas ferramentas e comparar resultados de diferentes fontes da literatura cinza e formal.

O \autoref{cap:fundamentacao} foca na apresentação dos conceitos fundamentais para o trabalho. O \autoref{cap:metodologia} descreve a metodologia e a escolha das fontes utilizadas. No \autoref{cap:literatura-cinzenta} é apresentada uma revisão da literatura cinza sobre o tema, já no \autoref{cap:literatura-formal}, a revisão foca na literatura formal. O \autoref{cap:analise} apresenta uma análise comparando e relacionando os trabalhos de literatura formal e cinzenta. Por fim, o \autoref{cap:conclusao} conclui o trabalho.