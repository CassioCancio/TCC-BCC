%!TeX root=../tese.tex

\chapter**{Introdução}\label{cap:introducao}

\section**{Contexto}

A engenharia de \emph{software} é um campo da computação que se dedica à produção e à manutenção de sistemas de \emph{software}. O termo foi consolidado em 1968 pela Organização do Tratado do Atlântico Norte (OTAN), no contexto da chamada ``crise do \emph{software}'', período marcado pelo aumento da complexidade dos sistemas computacionais e pela dificuldade em desenvolver programas de forma previsível e organizada. Desde então, a área passou por um processo contínuo de amadurecimento, com o surgimento de ferramentas, métodos e processos voltados à organização do desenvolvimento de \emph{software} e à viabilização de projetos cada vez mais complexos.

Nas décadas seguintes, avanços tecnológicos significativos contribuíram para transformar o cenário computacional, incluindo o aumento da capacidade de processamento, o barateamento do \emph{hardware}, tornando computadores e dispositivos móveis mais acessíveis, e a expansão da \emph{internet}, que possibilitou a inclusão de bilhões de pessoas em ambientes digitais. Esse processo resultou na geração de grandes volumes de dados sobre diferentes aspectos da vida cotidiana e virtual. 

Nesse contexto, técnicas de aprendizado de máquina puderam evoluir de forma acelerada, culminando no desenvolvimento de modelos capazes de aprender padrões complexos a partir de grandes conjuntos de dados. Entre esses avanços, destacam-se os modelos de inteligência artificial generativa, em especial os modelos de linguagem de grande escala (LLMs), capazes de gerar conteúdos novos, como texto e código, a partir de padrões aprendidos durante seu treinamento. 

Com o tempo, a aplicação de modelos de IA generativa em ferramentas usadas na engenharia de \emph{software} tornou-se cada vez mais popular. As ferramentas incluem \emph{chatbots} conversacionais utilizados para auxílio na escrita e compreensão de código, assistentes de programação integrados a ambientes de desenvolvimento integrados (IDEs), ferramentas para geração de testes e suporte à depuração. 

Nos últimos anos, diversos estudos passaram a investigar o uso dessas ferramentas no processo de desenvolvimento, analisando seus impactos, limitações e riscos, bem como propondo diretrizes para uma adoção segura e responsável (\cite{10705752} e \cite{10.1145/3715003}). Dados recentes indicam que a adoção dessas tecnologias já é expressiva entre desenvolvedores profissionais. Segundo uma pesquisa conduzida pela \citet{StackOverflow2025}, 80,7\% dos desenvolvedores profissionais participantes utilizam ferramentas de inteligência artificial em seu processo de desenvolvimento de \emph{software}, enquanto 4,6\% afirmam que pretendem adotá-las em breve. 

Este trabalho se propõe a investigar os impactos do uso de ferramentas de IA generativa na engenharia de \emph{software}, reunindo e analisando dados provenientes tanto da literatura formal quanto da literatura cinzenta. Considerando a ampla disseminação da IA na engenharia de \emph{software}, é razoável supor que seu uso produza impactos relevantes na produção de código e nas práticas de desenvolvimento, o que reforça a relevância deste trabalho. 

\section**{Objetivos}

Os objetivos deste trabalho são:

\begin{itemize}
    \item Reunir e sistematizar dados sobre os impactos do uso de ferramentas de IA generativa na produção de \emph{software}, considerando tanto evidências técnicas quanto percepções de profissionais da área;
    \item Analisar a evolução da percepção dos programadores em relação ao uso dessas ferramentas ao longo do tempo, identificando tendências, benefícios percebidos e preocupações recorrentes;
    \item Comparar e contrastar os resultados provenientes da literatura formal e da literatura cinzenta sobre o tema, com o objetivo de identificar convergências, divergências e lacunas entre pesquisas acadêmicas e evidências oriundas do mercado;
    \item Sintetizar os principais achados da análise, discutindo suas implicações para a prática do desenvolvimento de \emph{software} e para pesquisas futuras na área.
\end{itemize}

\section**{Estrutura do trabalho}
O \autoref{cap:fundamentacao} é uma fundamentação teórica, apresentando os conceitos essenciais para o trabalho. O \autoref{cap:selecao_estrategia} descreve a estratégia utilizada para a seleção das fontes e a organização das análises. No \autoref{cap:fontes}, são apresentadas análises de cada uma das fontes do trabalho, com um resumo dos resultados e da organização por trás das fontes. O \autoref{cap:analise} contém uma análise que compara e relaciona os estudos das duas frentes. Por fim, o \autoref{cap:conclusao} conclui o trabalho, apontando possíveis continuidades.