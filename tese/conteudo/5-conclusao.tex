%!TeX root=../tese.tex

\chapter{Conclusão}
\label{cap:conclusao}

O trabalho conclui, portanto, que o cenário provável para o futuro é de que a tendência de adoção de IA continue se expandindo pelas empresas. Os indícios até o momento sugerem que a qualidade dos códigos escritos na ``era da IA'' tem se deteriorado e que um dos principais usos feitos pelos programadores de ferramentas de IA é justamente a escrita de código.

De alguma maneira, os programadores parecem estar reagindo a essa piora na qualidade do código. A percepção dos profissionais sobre as ferramentas de IA tem ano após ano se convertendo numa visão mais sóbria e menos positiva. Essa mudança tem impactado os programadores nos seus usos de IA, de modo que tarefas de alta complexidade e responsabilidade tem sido menos confiadas à IA.

Este trabalho se baseou bastante em literaturas cinzentas para a sua análise e é importante que novas pesquisas sejam feitas no sentido de formalizar muito do que já aparece na literatura informal. Algumas comparações de pesquisas foram difíceis, inclusive porque não há um padrão muito bem definido até mesmo entre pesquisas de uma mesma empresa.

Outras análises relevantes seriam possíveis no sentido de continuar o acompanhamento das fontes apresentadas, já que muitas delas são anuais e poderão atualizar o parecer atual sobre percepções, impactos, métricas de código, entre outros. Além disso, com a adoção da IA agêntica em crescimento, é possível que novos impactos surjam, intensificando tendências já presentes em 2025. Este é outro foco relevante para a continuidade do tema.