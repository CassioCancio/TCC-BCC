\documentclass[
  xcolor={hyperref,svgnames,x11names,table},
  hyperref={pdfencoding=unicode,plainpages=false,pdfpagelabels=true,breaklinks=true},
  brazilian,english,12pt,aspectratio=149,
]{beamer}

\usepackage{imegoodies}
\usepackage[slides,hidelinks]{imelooks}

\graphicspath{{figuras/},{fig/},{logos/},{img/},{images/},{imagens/}}

\babeltags{br = brazilian, en = english}



\AtBeginSection[]{
  \begin{frame}<beamer>[t]{Sumário}
    \intermezzo
  \end{frame}
}

\newenvironment{coloredblock}[2]%
  {
    \setbeamercolor{block title}{fg=white,bg=#1!80!white}
    \setbeamercolor{block body}{fg=darkgray,bg=#1!20!white}
    \setbeamercolor{local structure}{fg=darkgray,bg=#1!20!white}
    \begin{block}{#2}
  }
  {\end{block}}

\addbibresource{bibliografia.bib}

\title[Percepções em Transformação]{Percepções em Transformação}

\subtitle{\normalsize Os impactos da IA generativa na produção de \emph{software}}

\author[Shortened Author Names]{Cássio Azevedo Cancio}

\institute{\textbf{Orientador:} Paulo Meirelles \\ \textbf{Coorientador:} Arthur Pilone}

\date{10 de dezembro de 2025}

\titlebgimage{\includegraphics[width=\paperwidth,height=\paperheight]%
             {bg-ime}}


\logos{
  \includegraphics[height=.9\baselineskip]{ime-logo}%
  \quad\quad
  \raisebox{.15\baselineskip}{\includegraphics[height=.6\baselineskip]{usp-logo}}%
  \par
  \hspace{.87\paperwidth}\includegraphics[width=.05\paperwidth]{cc-by}\par
}

\toccolumnsplit{5}

\qrcodeurl{https://cassiocancio.github.io/TCC-BCC/}

\begin{document}

\maketitle







































\section{Introdução}

\subsection{Contexto}

\begin{frame}{Contexto}
    \begin{itemize}
        \item A adoção de ferramentas de IA generativa para a engenharia de software é uma das grandes tendências da indústria.
        \item A pesquisa Stack Overflow Developer Survey aponta, desde 2023, o crescimento da adoção de ferramentas de IA na comunidade de desenvolvedores profissionais.
    \end{itemize}
\end{frame}

\subsection{Objetivo}

\begin{frame}{Objetivo}
    \begin{itemize}
        \item Levantar dados sobre os \textbf{impactos da IA generativa na produção de software}.
        \item Analisar a evolução da \textbf{percepção dos desenvolvedores} de acordo com pesquisas e relatórios.
        \item Comparar resultados de diferentes fontes da \textbf{literatura cinza e formal}.
    \end{itemize}
\end{frame}




% -------------------------
\section{Uso de IA em Engenharia de Software}

\begin{frame}{Uso de IA em 2025 (Stack Overflow)}
    \begin{center}
        \textbf{Uso de IA no desenvolvimento de software em 2025 por programadores profissionais}

        \vspace{0.5cm}
        \includegraphics[width=0.9\linewidth]{grafico-uso.png}
    \end{center}

    \vspace{0.3cm}
    Segundo previsões da Gartner:
    \begin{itemize}
        \item Em 2028, 90\% das empresas de software usarão \textbf{AI code assistants}.
        \item A participação da IA no ciclo de desenvolvimento passará de 5\% (2024) para 40\% (2027).
        \item Até 2027, 25\% dos bugs em produção virão de \textbf{código de IA não revisado}.
    \end{itemize}
\end{frame}

\begin{frame}{Casos de Uso da IA}
    De acordo com Sergeyuk et al. (2025), desenvolvedores preferem:
    \begin{itemize}
        \item \textbf{Delegar tarefas menos prazerosas} para ferramentas de IA.
        \item \textbf{Manter controle} das atividades mais interessantes.
    \end{itemize}

    \vspace{0.4cm}
    \scriptsize
    \begin{tabular}{p{4cm}p{3cm}p{3cm}}
        Atividade & Delegaria & Gosta de fazer \\
        \hline
        Escrita de testes           & 70\% & 30\% \\
        Documentação técnica        & 66\% & 26\% \\
        Criar novas features        & 27\% & 86\% \\
    \end{tabular}

    \vspace{0.4cm}
    \normalsize
    Segundo Stack Overflow 2024:
    \begin{itemize}
        \item 82\% usa IA para escrever código.
        \item 40,1\% para produzir documentação.
        \item 27,2\% para testar código.
    \end{itemize}
\end{frame}

\section{Resultados da GitClear}

\begin{frame}{Análise de Qualidade de Código (GitClear)}
    \textbf{Queda na refatoração:}
    \begin{itemize}
        \item Linhas movidas caíram de 24,1\% (2020) para 9,5\% (2024).
    \end{itemize}

    \vspace{0.3cm}
    \textbf{Aumento de duplicações:}
    \begin{itemize}
        \item 2024: linhas copiadas (12,3\%) superam as movidas (9,5\%).
        \item Commits com blocos duplicados cresceram de 0,7\% (2020) para 6,66\% (2024).
    \end{itemize}

    \vspace{0.3cm}
    \textbf{Aumento do retrabalho (churn):}
    \begin{itemize}
        \item Cresceu de 3,1\% (2020) para 5,7\% (2024).
    \end{itemize}

    \vspace{0.3cm}
    \textbf{Redução da durabilidade do código:}
    \begin{itemize}
        \item Linhas trocadas em <2 semanas: 60,4\% → 69,7\%.
        \item Linhas alteradas entre 4 semanas e 1 ano: 24,7\% → 16,9\%.
    \end{itemize}
\end{frame}

\begin{frame}{Tabela de Métricas (GitClear)}
    A GitClear analisou \textbf{211 milhões de linhas de código} alteradas entre 2020 e 2024.

    \vspace{0.4cm}
    \scriptsize
    \begin{center}
    \begin{tabular}{c|c|c|c|c|c}
        Ano & Adicionada & Movida & Copiada & Substituída & Churn \\
        \hline
        2020 & 39,2\% & 24,1\% & 8,3\% & 2,9\% & 3,1\% \\
        2021 & 39,5\% & 24,8\% & 8,4\% & 3,4\% & 3,3\% \\
        2022 & 40,9\% & 20,5\% & 9,4\% & 3,7\% & 3,3\% \\
        2023 & 42,3\% & 15,8\% & 10,6\% & 3,6\% & 4,5\% \\
        2024 & 46,2\% & 9,5\%  & 12,3\% & 4,2\% & 5,7\% \\
    \end{tabular}
    \end{center}
    \normalsize
\end{frame}












































\section{Referências}

\begin{frame}[allowframebreaks]{Referências}
  \nocite{anssi_2024,StackOverflow2023,StackOverflow2024,StackOverflow2025,elsevier,gitclear2025,gartner_ai_code_assistants_2024, gartner2024agenticAI}
  \printbibliography
\end{frame} 
    
\begin{frame}[t]{\insertshorttitle}
  \overview

  \vspace{1\baselineskip}
  \bgroup
  \centering
  \egroup

\end{frame}

\end{document}
