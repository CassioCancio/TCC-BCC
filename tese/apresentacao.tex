% Authors: Nelson Lago and Fernanda Magano
% This file is distributed under the MIT Licence

%%%%%%%%%%%%%%%%%%%%%%%%%%%%%%%%%%%%%%%%%%%%%%%%%%%%%%%%%%%%%%%%%%%%%%%%%%%%%%%%
%%%%%%%%%%%%%%%%%%%%%%%%%%%%%%%%% PREÂMBULO %%%%%%%%%%%%%%%%%%%%%%%%%%%%%%%%%%%%
%%%%%%%%%%%%%%%%%%%%%%%%%%%%%%%%%%%%%%%%%%%%%%%%%%%%%%%%%%%%%%%%%%%%%%%%%%%%%%%%

% aspectratio default é 4:3;
% as mais úteis são 169 (16:9), 1610 (16:10) e 149 (14:9)
% A língua padrão é a última citada
\documentclass[
  xcolor={hyperref,svgnames,x11names,table},
  hyperref={pdfencoding=unicode,plainpages=false,pdfpagelabels=true,breaklinks=true},
  brazilian,english,12pt,aspectratio=149,
]{beamer}

% Vários pacotes e opções de configuração genéricos
\usepackage{imegoodies}
\usepackage[slides,hidelinks]{imelooks}

% Diretórios onde estão as figuras; com isso, não é necessário (mas
% é permitido) colocar o caminho completo em \includegraphics. Note
% que a extensão nunca é necessária (mas é permitida), ou seja, o
% resultado é o mesmo com "\includegraphics{figuras/foto.jpeg}",
% "\includegraphics{foto.jpeg}", "\includegraphics{figuras/foto}"
% ou "\includegraphics{foto}".
\graphicspath{{figuras/},{fig/},{logos/},{img/},{images/},{imagens/}}

% Comandos rápidos para mudar de língua:
% \en -> muda para o inglês
% \br -> muda para o português
% \texten{blah} -> o texto "blah" é em inglês
% \textbr{blah} -> o texto "blah" é em português
\babeltags{br = brazilian, en = english}


%%%%%%%%%%%%%%%%%%%%%%%%%% COMANDOS PARA O USUÁRIO %%%%%%%%%%%%%%%%%%%%%%%%%%%%%

% A cada nova seção, recapitula o sumário.
% Para desabilitar, é só comentar este trecho
\AtBeginSection[]{
  \begin{frame}<beamer>[t]{Sumário}
    \intermezzo
  \end{frame}
}

% Blocos de cor personalizada
\newenvironment{coloredblock}[2]%
  {
    \setbeamercolor{block title}{fg=white,bg=#1!80!white}
    \setbeamercolor{block body}{fg=darkgray,bg=#1!20!white}
    \setbeamercolor{local structure}{fg=darkgray,bg=#1!20!white}
    \begin{block}{#2}
  }
  {\end{block}}


%%%%%%%%%%%%%%%%%%%%%%%%%%%%%%%%%%%%%%%%%%%%%%%%%%%%%%%%%%%%%%%%%%%%%%%%%%%%%%%%
%%%%%%%%%%%%%%%%%%%%%%%%%% METADADOS E SLIDE TÍTULO %%%%%%%%%%%%%%%%%%%%%%%%%%%%
%%%%%%%%%%%%%%%%%%%%%%%%%%%%%%%%%%%%%%%%%%%%%%%%%%%%%%%%%%%%%%%%%%%%%%%%%%%%%%%%

% O arquivo com os dados bibliográficos para biblatex; você pode usar
% este comando mais de uma vez para acrescentar múltiplos arquivos
\addbibresource{bibliografia.bib}

% Este comando permite acrescentar itens à lista de referências sem incluir
% uma referência de fato no texto (pode ser usado em qualquer lugar do texto)
%\nocite{bronevetsky02,schmidt03:MSc, FSF:GNU-GPL, CORBA:spec, MenaChalco08}
% Com este comando, todos os itens do arquivo .bib são incluídos na lista
% de referências
%\nocite{*}

\title[Percepções em Transformação]{Percepções em Transformação}

\subtitle{\normalsize Os impactos da IA generativa na produção de \emph{software}}

\author[Shortened Author Names]{Cássio Azevedo Cancio}

\institute{\textbf{Orientador:} Paulo Meirelles \\ \textbf{Coorientador:} Arthur Pilone}

\date{10 de dezembro de 2025}

% Coloca a imagem no fundo da página de título
\titlebgimage{\includegraphics[width=\paperwidth,height=\paperheight]%
             {bg-ime}}

% Logotipos no rodapé da página de título. Na verdade, qualquer coisa pode
% ser colocada aqui para aparecer próximo ao final da página de título.
% Esse material é colocado em uma minipage de largura .7\textwidth, porque
% o pano de fundo tem um pedaço muito escuro à direita da página.
%
% De acordo com o manual de identidade visual, a altura do logo da USP deve
% ser 2/3 da altura do logo do IME; o logo da USP deve ficar deslocado para
% cima 1/6 da altura do logo do IME. As posições e tamanhos dos logos da
% fapesp/capes/cnpq não seguem nenhuma lógica bem definida (na verdade,
% nada está alinhado com nada!): eu (Nelson) fiz "a olho" da maneira que
% me pareceu mais harmônica.
\logos{%
  \includegraphics[height=.9\baselineskip]{ime-logo}%
  \quad\quad
  \raisebox{.15\baselineskip}{\includegraphics[height=.6\baselineskip]{usp-logo}}%
  %\quad\quad
  %\raisebox{.178\baselineskip}{\includegraphics[height=.515\baselineskip]{fapesp-logo}}%
  %\quad\quad
  %\raisebox{-.1732\baselineskip}{\includegraphics[height=1.2\baselineskip]{capes-logo}}%
  %\quad\quad
  %\raisebox{.124\baselineskip}{\includegraphics[height=.6\baselineskip]{cnpq-logo}}%
  \par
  % Queremos que o logo cc-by fique próximo da margem direita da página, mas
  % este material está dentro de uma minipage de largura .7\textwidth. Assim,
  % usamos esse hspace, que "estoura" a largura da minipage para a direita.
  \hspace{.87\paperwidth}\includegraphics[width=.05\paperwidth]{cc-by}\par
}

% Inclui ou não o qrcode no sumário da apresentação
%\qrcodeintoc % default
%\noqrcodeintoc

% O slide de sumário pode ser dividido em colunas; o parâmetro
% determina após qual o número da seção fazer a quebra de coluna
% (use zero para uma coluna ou simplesmente omita este comando).
\toccolumnsplit{5}


%%%%%%%%%%%%%%%%%%%%%%%%%%%%%%%%%%%%%%%%%%%%%%%%%%%%%%%%%%%%%%%%%%%%%%%%%%%%%%%%
%%%%%%%%%%%%%%%%%%%%%%%%%%%% INÍCIO DA APRESENTAÇÃO %%%%%%%%%%%%%%%%%%%%%%%%%%%%
%%%%%%%%%%%%%%%%%%%%%%%%%%%%%%%%%%%%%%%%%%%%%%%%%%%%%%%%%%%%%%%%%%%%%%%%%%%%%%%%
\qrcodeurl{https://cassiocancio.github.io/TCC-BCC/}

\begin{document}

\maketitle

\begin{frame}[t]{Sumário}
  \overview
\end{frame}

\section{Introdução}

\begin{frame}{Contexto}
  \begin{itemize}
    \item O compromisso de direitos autorais buscava equilibrar interesses públicos e privados
    \item Hoje, mudanças na legislação e avanços tecnológicos praticamente destruíram esse equilíbrio
    \item[]\strut % https://github.com/schlcht/microtype/issues/6
    \item Em reação, surgiu o movimento do software livre
          \begin{itemize}
            \item Retorno ao compartilhamento (do código-fonte) e à colaboração (troca de ideias e trabalho em equipe)
            \item Formalização com o projeto GNU
            \item Só é realmente possível quando existem condições favoráveis para a troca de código-fonte
                  \begin{itemize}
                    \item conforme evidenciado pelo crescimento que acompanhou o boom da Internet
                  \end{itemize}
          \end{itemize}
  \end{itemize}

\end{frame}

\begin{frame}[plain]
  % Em um poster ou apresentação, normalmente não é necessário usar
  % \begin{figure} ou \begin{table}, basta usar \includegraphics ou
  % \begin{tabular}. \begin{figure} e \begin{table} só são necessários
  % se você quiser acrescentar legendas ou usar subfiguras. Nesses casos,
  % [H] é obrigatório com beamer e com tcolorbox. Uma outra opção para
  % inserir legendas é usar \captionof.
  \begin{figure}[H]
    \includegraphics[width=.7\textwidth]{ccsl-logo}
    \caption*{Logo do CCSL} % Com "*", suprime a numeração
  \end{figure}
\end{frame}

\begin{frame}[standout]
  Isto é um problema!
\end{frame}

\section{Conceitos}

\begin{frame}{Conceitos}

  A Wikipedia não é uma boa fonte para pesquisa acadêmica,\\
  mas, ainda assim, é útil. A entrada sobre pangramas afirma:

  \vspace{\baselineskip}

  \begin{block}{O que são pangramas?}
    \begin{itemize}
      \item Um \alert{pangrama} é uma frase que utiliza cada letra de um dado
            alfabeto pelo menos uma vez.
      \item Pangramas têm sido usados para exibir fontes, testar equipamentos,
            e desenvolver habilidades em caligrafia, escrita à mão e digitação.
    \end{itemize}
  \end{block}

  \vspace{\baselineskip}

  (\url{https://en.wikipedia.org/wiki/Pangram})

\end{frame}

\begin{frame}{Pangramas -- exemplos}
  \begin{columns}[t]

    \column{.5\textwidth}
    \begin{coloredblock}{red!90!black}{Alguns pangramas em inglês}
      \begin{itemize}
        \item A quick brown fox jumps over the lazy dog
        \item Sphinx of black quartz, judge my vow
        \item How vexingly quick daft zebras jump
        \item Pack my box with five dozen liquor jugs
      \end{itemize}
    \end{coloredblock}

    \column{.5\textwidth}
    \begin{coloredblock}{red!90!black}{Alguns pangramas em português}
      \begin{itemize}
        \item Vejo xá gritando que fez show sem playback
        \item Vi que ex-janota gordo fez show com playback
        \item Já fiz vinho com toque de kiwi para belga sexy
        \item Vejo galã sexy pôr quinze kiwis à força em baú achatado
      \end{itemize}
    \end{coloredblock}

  \end{columns}
\end{frame}

\begin{frame}{Teoremas e demonstrações}
  \pause
  \begin{theorem}[Um teorema de exemplo]
    Teorema\dots
  \end{theorem}

  \pause
  \begin{example}[Um exemplo de exemplo]
    Exemplo\dots
  \end{example}

  \pause
  \begin{proof}[Uma demonstração de exemplo]
    Demonstração\dots
  \end{proof}

  \pause
  \begin{definition}[Uma definição de exemplo]
    Definição\dots
  \end{definition}

  \pause
  \begin{proposition}[Uma proposição de exemplo]
    Proposição\dots
  \end{proposition}
\end{frame}

\section{Trabalhos Relacionados}

\begin{frame}{Trabalhos Relacionados}

  \begin{table}[H] % Veja comentário mais acima sobre [H]
    \centering
    \singlespacing\vspace{-\baselineskip}
    \begin{tabular}{ccl}
      \toprule
      Code       & Abbreviation & \makecell{Full \\Name} \\
      \midrule
      \texttt{A} & Ala          & Alanine        \\
      \texttt{C} & Cys          & Cysteine       \\
      ...        & ...          & ...            \\
      \texttt{W} & Trp          & Tryptophan     \\
      \texttt{Y} & Tyr          & Tyrosine       \\
      \bottomrule
    \end{tabular}
    \caption*{Uma tabela inútil.} % Com "*", suprime a numeração
  \end{table}

\end{frame}

\section{Metodologia}

\begin{frame}{Metodologia}
\end{frame}

\section{Resultados}

\subsection{Validação e Análise}

\begin{frame}{Validação}
\end{frame}

\begin{frame}{Estudo de Caso}
\end{frame}

\section{Conclusão e Trabalhos Futuros}

\begin{frame}{Conclusão e Trabalhos Futuros}
\end{frame}

\section{Referências}

\begin{frame}[allowframebreaks]{Referências}
  \footnotesize % diminui o tamanho da fonte
  \nocite{anssi_2024,StackOverflow2023,StackOverflow2024,StackOverflow2025,elsevier,gitclear2025,gartner_ai_code_assistants_2024, gartner2024agenticAI}
  \printbibliography
\end{frame} 
    
% Slides do tipo "Fim" ou "Perguntas?" não são muito úteis; ao invés
% disso, é mais interessante definir um slide final recapitulando o
% que foi visto.
\begin{frame}[t]{\insertshorttitle}
  \overview

  % \begin{center} acrescenta espaço vertical; como possivelmente temos
  % pouco espaço aqui, vamos usar centering e adicionar o espaço manualmente
  \vspace{1\baselineskip}
  \bgroup
  \centering
  \egroup

\end{frame}

\end{document}
