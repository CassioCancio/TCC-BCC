% Author: Nelson Lago
% This file is distributed under the MIT Licence

%%%%%%%%%%%%%%%%%%%%%%%%%%%%%%%%%%%%%%%%%%%%%%%%%%%%%%%%%%%%%%%%%%%%%%%%%%%%%%%%
%%%%%%%%%%%%%%%%%%%%%%%%%%%%%%%%% PREÂMBULO %%%%%%%%%%%%%%%%%%%%%%%%%%%%%%%%%%%%
%%%%%%%%%%%%%%%%%%%%%%%%%%%%%%%%%%%%%%%%%%%%%%%%%%%%%%%%%%%%%%%%%%%%%%%%%%%%%%%%

% A língua padrão é a última da lista
\documentclass[a1paper,brazilian,english]{article}

% Vários pacotes e opções de configuração genéricos
\usepackage{imegoodies}
\usepackage[poster,hidelinks]{imelooks}
% \tcbposterset{fontsize = 32pt} % default, mude se necessário

% Diretórios onde estão as figuras; com isso, não é necessário (mas
% é permitido) colocar o caminho completo em \includegraphics. Note
% que a extensão nunca é necessária (mas é permitida), ou seja, o
% resultado é o mesmo com "\includegraphics{figuras/foto.jpeg}",
% "\includegraphics{foto.jpeg}", "\includegraphics{figuras/foto}"
% ou "\includegraphics{foto}".
\graphicspath{{figuras/},{fig/},{logos/},{img/},{images/},{imagens/}}

% Comandos rápidos para mudar de língua:
% \en -> muda para o inglês
% \br -> muda para o português
% \texten{blah} -> o texto "blah" é em inglês
% \textbr{blah} -> o texto "blah" é em português
\babeltags{br = brazilian, en = english}


%%%%%%%%%%%%%%%%%%%%%%%%%%%%%%%%%%%%%%%%%%%%%%%%%%%%%%%%%%%%%%%%%%%%%%%%%%%%%%%%
%%%%%%%%%%%%%%%%%%%%%%%%%%%%%%%%%% METADADOS %%%%%%%%%%%%%%%%%%%%%%%%%%%%%%%%%%%
%%%%%%%%%%%%%%%%%%%%%%%%%%%%%%%%%%%%%%%%%%%%%%%%%%%%%%%%%%%%%%%%%%%%%%%%%%%%%%%%

% O arquivo com os dados bibliográficos para biblatex; você pode usar
% este comando mais de uma vez para acrescentar múltiplos arquivos
\addbibresource{bibliografia.bib}

% Este comando permite acrescentar itens à lista de referências sem incluir
% uma referência de fato no texto (pode ser usado em qualquer lugar do texto)
%\nocite{bronevetsky02,schmidt03:MSc, FSF:GNU-GPL, CORBA:spec, MenaChalco08}
% Com este comando, todos os itens do arquivo .bib são incluídos na lista
% de referências
%\nocite{*}


%%%%%%%%%%%%%%%%%%%%%%%%%%%%%%%%%%%%%%%%%%%%%%%%%%%%%%%%%%%%%%%%%%%%%%%%%%%%%%%%
%%%%%%%%%%%%%%%%%%%%%%%%%%%%%%% INÍCIO DO POSTER %%%%%%%%%%%%%%%%%%%%%%%%%%%%%%%
%%%%%%%%%%%%%%%%%%%%%%%%%%%%%%%%%%%%%%%%%%%%%%%%%%%%%%%%%%%%%%%%%%%%%%%%%%%%%%%%


% Existem várias packages para criar pôsteres com LaTeX (a0poster, baposter,
% tikzposter, sciposter...). As mais comuns atualmente são beamerposter
% e tcolorbox (com sua biblioteca "poster"). Ambas funcionam muito bem;
% beamerposter é mais familiar (ela simplesmente utiliza beamer com alguns
% ajustes no tamanho das fontes e do papel), mas com tcolorbox o alinhamento
% vertical dos elementos é MUITO mais simples, e esta é a solução adotada
% aqui. Vale muito a pena ler a documentação com "texdoc tcolorbox" e
% "texdoc tcolorbox-tutorial-poster".

% Um pôster com tcolorbox é composto por blocos (posterboxes) coloridos
% de tamanho variável; cada bloco pode conter textos ou imagens e um
% título opcional. O pôster utiliza uma grade de dimensões definidas em
% \begin{tcposter} com "rows=" e "columns=" para fazer o alinhamento:
% para cada posterbox, podemos dizer "row=X, column=Y" para definir sua
% posição. Além disso, podemos dizer "span=A, rowspan=B" para fixar
% seu tamanho. Sem "span" e "rowspan", uma posterbox tem pelo menos o
% tamanho de uma célula da grade, mas se seu tamanho natural for maior
% ela extrapola esse tamanho. "span" e "rowspan" podem ser números
% não-inteiros (como 0.8 ou 1.4).
%
% "\begin{posterbox}" recebe um conjunto de parâmetros opcional e um
% conjunto de parâmetros obrigatório:
%
% "\begin{posterbox}[opcional]{obrigatório}".
%
% O conjunto de parâmetros opcional é onde inserimos os parâmetros comuns
% de tcolorbox, como "adjusted title", "coltext", "titlerule" etc.; o
% conjunto de parâmetros obrigatório é usado para determinar as dimensões
% e a posição da posterbox, ou seja, as opções "name", "column", "below",
% "span" etc.
%
% ALINHAMENTO HORIZONTAL
%
% É possível definir um poster com 2 colunas e fazer algo como
%
% \posterbox{column=1, span=1.3}{blah}
% \posterbox{column*=2, span=0.7}{blah}
%
% A segunda posterbox será alinhada à direita ("column*="), então as
% duas serão colocadas lado-a-lado sem sobreposições.
%
% Na prática, no entanto, é mais fácil fazer como no exemplo abaixo:
% definimos que o poster tem 12 colunas, o que nos permite dividir
% sua largura em 2, 3, 4 ou 6 colunas iguais ou diferentes (como
% 1/2 + 1/2, 2/3 + 1/3, 1/4 + 1/4 + 1/2, 1/4 + 1/6 + 1/4 + 1/3 etc).
%
% ALINHAMENTO VERTICAL
%
% Embora seja possível alinhar as posterboxes em função da grade na
% vertical, uma outra possibilidade é utilizar "above", "below" e
% "between", como no exemplo abaixo: basta associar um nome "blah" a
% uma determinada posterbox e, em outra, dizer "below=blah". Lembre-se
% que a posterbox de nome "blah" deve ser definida *antes* que outra
% possa fazer referência a ela. Também é possível fazer "below=top",
% "above=bottom" etc. A opção "equal height group" também é muito útil.
% Nada impede que você use estratégias de alinhamento diferentes para
% cada posterbox.

% Este modelo define a opção "smallmargins", que diminui a distância
% entre o conteúdo de uma posterbox e suas bordas. Use com parcimônia!

\begin{document}

% Em um poster não há \maketitle

\begin{tcbposter}[
        poster = {
                %showframe, % muito útil durante a preparação do poster
                rows = 6,
                columns = 12,
                colspacing = 1.2cm,
                rowspacing = .8cm,
            },
    ]

\posterbox[titlebox]{name=titlebox, below=top, column=1, span=12}{

    \begin{minipage}{0.85\linewidth} % largura da coluna do texto
        {\Large IA Generativa na Engenharia de Software \\}
        \vspace{0.25cm}
        \large Impactos da IA na produção de código e a evolução da percepção sobre a IA \\
        \vspace{0.75cm}
        \normalsize

        Autor: Cássio Cancio | Orientadores: Profº Paulo Meirelles e Arthur Pilone
    \end{minipage}
    \hfill
    \begin{minipage}{0.1\linewidth} % largura da coluna da imagem
        \includegraphics[width=\linewidth]{vertical-simplificada}
    \end{minipage}

}


    \posterbox[footerbox]{name=footerbox, above=bottom, column=1, span=12}{
    }

    \posterbox[adjusted title = Contexto e Motivação]
    {name=intro, below=titlebox, column=1, span=12}{

        A adoção de assistentes de código baseados em IA (LLMs) transformou o desenvolvimento de software.
        Segundo o Gartner, a previsão é que \textbf{90\% dos engenheiros de software} utilizem assistentes de IA até 2028.
        No entanto, apesar do aumento na produtividade percebida, surgem evidências de impactos negativos na qualidade do código e novos vetores de risco.
        Este trabalho analisa dados de 211 milhões de linhas de código e relatórios da indústria para traçar o cenário atual e futuro.

    }

    %%%%%%%% Quatro colunas com "equal height group" %%%%%%%%

    \posterbox[adjusted title = Percepção (Stack Overflow),
        smallmargins, equal height group = quartos]
    {name=so, below=intro, column=1, span=3}{

        \begin{itemize}
            \item \textbf{72\%} dos desenvolvedores são favoráveis ao uso de IA[cite: 14].
            \item \textbf{Paradoxo da Confiança:} Apenas 43\% confiam na precisão da IA, enquanto 30\% são céticos[cite: 24].
            \item 63\% já utilizam IA no processo de desenvolvimento[cite: 2161].
        \end{itemize}
    }

    \posterbox[adjusted title = Delegação de Tarefas (Elsevier),
        smallmargins, equal height group = quartos]
    {name=elsevier, below=intro, column=4, span=3}{

        Desenvolvedores preferem delegar tarefas "chatas" e manter o controle sobre as criativas:
        \begin{description}
            \item[Alta Delegação:] Escrita de Testes (70\%), Documentação e Artefatos de Linguagem Natural.
            \item[Baixa Delegação:] Implementação de novas \textit{features} (apenas 18\% delegariam totalmente).
        \end{description}

    }

    \posterbox[adjusted title = Futuro: IA Agêntica (Gartner),
        smallmargins, equal height group = quartos]
    {name=gartner, below=intro, column=7, span=3}{

        Transição de "Chatbots" para \textbf{Agentes Autônomos}:
        \begin{itemize}
            \item \textbf{2028:} 33\% dos softwares empresariais terão IA agêntica (vs <1\% em 2024).
            \item \textbf{Autonomia:} Agentes tomarão 15\% das decisões diárias de trabalho sem intervenção humana.
        \end{itemize}
    }

    \posterbox[adjusted title = Riscos de Segurança (ANSSI/BSI),
        smallmargins, equal height group = dados]
    {name=riscos, below=intro, column=10, span=3}{

        \begin{itemize}
            \item \textbf{Alucinação de Pacotes:} Recomendação de bibliotecas inexistentes que podem ser exploradas por atacantes.
            \item \textbf{Shadow AI:} Uso de ferramentas não homologadas expondo dados sensíveis.
            \item \textbf{Viés de Automação:} Excesso de confiança em código inseguro.
        \end{itemize}
    }

    %%%%%%%% Duas colunas %%%%%%%%

    % Como temos 2 caixas à esquerda e uma caixa à direita, não podemos
    % simplesmente usar "equal height group" aqui, então definimos
    % manualmente a altura das caixas de maneira que as duas colunas
    % tenham o mesmo tamanho.

    %%% Esquerda
    \posterbox[adjusted title = Análise de Qualidade de Código (GitClear)]
    {name=gitclearText, below=so,
        column=1, span=6, rowspan=1.3}{

        Análise de \textbf{211 milhões de linhas de código} alteradas entre 2020 e 2024 revela tendências preocupantes na manutenibilidade 2172].

        \textbf{1. Queda na Refatoração (Moved Lines):}
        \begin{itemize}
            \item Linhas movidas (indicador de refatoração/reuso) caíram de 24,1\% (2020) para \textbf{9,5\%} (2024).
            \item Menos reuso implica em violação do princípio DRY (\textit{Don't Repeat Yourself}).
        \end{itemize}

        \vspace{0.5cm}

        \textbf{2. Aumento da Duplicação (Copy/Paste):}
        \begin{itemize}
            \item Em 2024, pela primeira vez, a % de linhas copiadas (\textbf{12,3\%}) superou as linhas movidas 2257].
            \item Ocorrência de blocos de código duplicados cresceu de 0,7\% (2020) para \textbf{6,66\%} (2024).
        \end{itemize}

        \vspace{0.5cm}

        \textbf{3. Aumento do Churn (Retrabalho):}
        \begin{itemize}
            \item O \textit{Churn} (código revertido em <2 semanas) cresceu para 5,7\%, indicando código gerado rapidamente mas com baixa durabilidade.
        \end{itemize}
    }

    \posterbox[adjusted title = Visualização das Métricas (GitClear)]
    {name=gitclearVisual, below=so,
        column=7, span=6, rowspan=1.3}{

        \centering
        \includegraphics[width=0.9\textwidth, height=9cm, keepaspectratio]{example-image-a}

        \vspace{0.5cm}
        \textit{Figura 1: Interseção histórica em 2024 onde a taxa de código copiado (vermelho) ultrapassa a refatoração (azul), sugerindo débito técnico futuro.}

        \vspace{1cm}

        \begin{tcolorbox}[colback=imesoftblue!10!white, title=Conclusão da Análise]
            O uso de IA incentiva a adição rápida de novo código ("Fire and Forget") em detrimento da manutenção e estruturação a longo prazo. Isso resulta em bases de código inchadas e mais propensas a bugs 2439].
        \end{tcolorbox}
    }

    %%% Direita
    \posterbox[adjusted title = Referências Selecionadas, smallmargins]
    {name=bibbox, between=gitclearText and footerbox, column=1, span=8}{
        \footnotesize
        \begin{itemize}
            \item \textbf{[GitClear 2025]} Harding, W. "AI Copilot Code Quality: Evaluating 2024's Increased Defect Rate".
            \item \textbf{[Gartner 2024]} "Principais Tendências Tecnológicas Estratégicas para 2025: IA Agêntica".
            \item \textbf{[Gartner MQ]} "Magic Quadrant for AI Code Assistants", 2024.
            \item \textbf{[Stack Overflow]} "2024 Developer Survey".
            \item \textbf{[Elsevier 2025]} Sergeyuk, A. et al. "Using AI-based coding assistants in practice".
            \item \textbf{[ANSSI/BSI]} "AI Coding Assistants - Security Recommendations", 2024.
        \end{itemize}}

    %%%%%%% Colunas assimétricas (2/3 + 1/3) %%%%%%%%

    %%% Esquerda. Como usamos "between", a altura desta caixa
    %             é definida pela posição das outras duas.
    \posterbox[adjusted title = Resumo Estatístico 2024,
        colframe = imeyellow, smallmargins]
    {between=gitclearVisual and footerbox, column=9, span=4}{

        \centering
        \small
        \begin{tabular}{lcc}
            \toprule
            Métrica            & 2020   & 2024                        \\
            \midrule
            Linhas Movidas     & 24.1\% & \textbf{9.5\%} $\downarrow$ \\
            Duplicação         & 0.7\%  & \textbf{6.6\%} $\uparrow$   \\
            Churn (Retrabalho) & 3.1\%  & \textbf{5.7\%} $\uparrow$   \\
            Adoção de IA       & <5\%   & \textbf{>60\%} $\uparrow$   \\
            \bottomrule
        \end{tabular}

        \vspace{0.2cm}
        \footnotesize{Fontes: GitClear  e Stack Overflow.}

    }
\end{tcbposter}

\end{document}
