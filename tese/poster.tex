% Author: Nelson Lago
% This file is distributed under the MIT Licence

%%%%%%%%%%%%%%%%%%%%%%%%%%%%%%%%%%%%%%%%%%%%%%%%%%%%%%%%%%%%%%%%%%%%%%%%%%%%%%%%
%%%%%%%%%%%%%%%%%%%%%%%%%%%%%%%%% PREÂMBULO %%%%%%%%%%%%%%%%%%%%%%%%%%%%%%%%%%%%
%%%%%%%%%%%%%%%%%%%%%%%%%%%%%%%%%%%%%%%%%%%%%%%%%%%%%%%%%%%%%%%%%%%%%%%%%%%%%%%% 

% A língua padrão é a última da lista
\documentclass[a1paper,english,brazilian]{article}

% Vários pacotes e opções de configuração genéricos
\usepackage{imegoodies}
\usepackage[poster,hidelinks]{imelooks}
\tcbposterset{fontsize = 28pt} % default, mude se necessário
\usepackage{pgfplots}
\pgfplotsset{compat=1.18}
\usepackage{pgf-pie}
\usepackage{qrcode} 

% Diretórios onde estão as figuras; com isso, não é necessário (mas
% é permitido) colocar o caminho completo em \includegraphics. Note
% que a extensão nunca é necessária (mas é permitida), ou seja, o
% resultado é o mesmo com "\includegraphics{figuras/foto.jpeg}",
% "\includegraphics{foto.jpeg}", "\includegraphics{figuras/foto}"
% ou "\includegraphics{foto}".
\graphicspath{{figuras/},{fig/},{logos/},{img/},{images/},{imagens/}}

% Comandos rápidos para mudar de língua:
% \en -> muda para o inglês
% \br -> muda para o português
% \texten{blah} -> o texto "blah" é em inglês
% \textbr{blah} -> o texto "blah" é em português
\babeltags{br = brazilian, en = english}


%%%%%%%%%%%%%%%%%%%%%%%%%%%%%%%%%%%%%%%%%%%%%%%%%%%%%%%%%%%%%%%%%%%%%%%%%%%%%%%%
%%%%%%%%%%%%%%%%%%%%%%%%%%%%%%%%%% METADADOS %%%%%%%%%%%%%%%%%%%%%%%%%%%%%%%%%%%
%%%%%%%%%%%%%%%%%%%%%%%%%%%%%%%%%%%%%%%%%%%%%%%%%%%%%%%%%%%%%%%%%%%%%%%%%%%%%%%%

% O arquivo com os dados bibliográficos para biblatex; você pode usar 
% este comando mais de uma vez para acrescentar múltiplos arquivos 
\addbibresource{bibliografia.bib}

% Este comando permite acrescentar itens à lista de referências sem incluir
% uma referência de fato no texto (pode ser usado em qualquer lugar do texto)
%\nocite{bronevetsky02,schmidt03:MSc, FSF:GNU-GPL, CORBA:spec, MenaChalco08}
% Com este comando, todos os itens do arquivo .bib são incluídos na lista
% de referências
%\nocite{*}


%%%%%%%%%%%%%%%%%%%%%%%%%%%%%%%%%%%%%%%%%%%%%%%%%%%%%%%%%%%%%%%%%%%%%%%%%%%%%%%%
%%%%%%%%%%%%%%%%%%%%%%%%%%%%%%% INÍCIO DO POSTER %%%%%%%%%%%%%%%%%%%%%%%%%%%%%%%
%%%%%%%%%%%%%%%%%%%%%%%%%%%%%%%%%%%%%%%%%%%%%%%%%%%%%%%%%%%%%%%%%%%%%%%%%%%%%%%%


% Existem várias packages para criar pôsteres com LaTeX (a0poster, baposter,
% tikzposter, sciposter...). As mais comuns atualmente são beamerposter
% e tcolorbox (com sua biblioteca "poster"). Ambas funcionam muito bem;
% beamerposter é mais familiar (ela simplesmente utiliza beamer com alguns
% ajustes no tamanho das fontes e do papel), mas com tcolorbox o alinhamento
% vertical dos elementos é MUITO mais simples, e esta é a solução adotada
% aqui. Vale muito a pena ler a documentação com "texdoc tcolorbox" e
% "texdoc tcolorbox-tutorial-poster".

% Um pôster com tcolorbox é composto por blocos (posterboxes) coloridos
% de tamanho variável; cada bloco pode conter textos ou imagens e um
% título opcional. O pôster utiliza uma grade de dimensões definidas em
% \begin{tcposter} com "rows=" e "columns=" para fazer o alinhamento:
% para cada posterbox, podemos dizer "row=X, column=Y" para definir sua
% posição. Além disso, podemos dizer "span=A, rowspan=B" para fixar
% seu tamanho. Sem "span" e "rowspan", uma posterbox tem pelo menos o
% tamanho de uma célula da grade, mas se seu tamanho natural for maior
% ela extrapola esse tamanho. "span" e "rowspan" podem ser números
% não-inteiros (como 0.8 ou 1.4).
%
% "\begin{posterbox}" recebe um conjunto de parâmetros opcional e um
% conjunto de parâmetros obrigatório:
%
% "\begin{posterbox}[opcional]{obrigatório}".
%
% O conjunto de parâmetros opcional é onde inserimos os parâmetros comuns
% de tcolorbox, como "adjusted title", "coltext", "titlerule" etc.; o
% conjunto de parâmetros obrigatório é usado para determinar as dimensões
% e a posição da posterbox, ou seja, as opções "name", "column", "below",
% "span" etc.
%
% ALINHAMENTO HORIZONTAL
%
% É possível definir um poster com 2 colunas e fazer algo como
%
% \posterbox{column=1, span=1.3}{blah}
% \posterbox{column*=2, span=0.7}{blah}
%
% A segunda posterbox será alinhada à direita ("column*="), então as
% duas serão colocadas lado-a-lado sem sobreposições.
%
% Na prática, no entanto, é mais fácil fazer como no exemplo abaixo:
% definimos que o poster tem 12 colunas, o que nos permite dividir
% sua largura em 2, 3, 4 ou 6 colunas iguais ou diferentes (como
% 1/2 + 1/2, 2/3 + 1/3, 1/4 + 1/4 + 1/2, 1/4 + 1/6 + 1/4 + 1/3 etc).
%
% ALINHAMENTO VERTICAL
%
% Embora seja possível alinhar as posterboxes em função da grade na
% vertical, uma outra possibilidade é utilizar "above", "below" e
% "between", como no exemplo abaixo: basta associar um nome "blah" a
% uma determinada posterbox e, em outra, dizer "below=blah". Lembre-se
% que a posterbox de nome "blah" deve ser definida *antes* que outra
% possa fazer referência a ela. Também é possível fazer "below=top",
% "above=bottom" etc. A opção "equal height group" também é muito útil.
% Nada impede que você use estratégias de alinhamento diferentes para
% cada posterbox.

% Este modelo define a opção "smallmargins", que diminui a distância
% entre o conteúdo de uma posterbox e suas bordas. Use com parcimônia!

\begin{document}
% Em um poster não há \maketitle

\begin{tcbposter}[
        poster = {
                % showframe, % muito útil durante a preparação do poster
                rows = 5,
                columns = 12,
                colspacing = 1cm,
                rowspacing = .8cm,
            },
    ]

    \posterbox[titlebox]{name=titlebox, below=top, column=1, span=12}{

        \begin{minipage}{0.85\linewidth} % largura da coluna do texto
            {\Large Percepções em Transformação \\}
            \vspace{0.25cm}
            \large Os Impactos da IA Generativa na Produção de Software \\
            \vspace{0.75cm}
            \normalsize

            Autor: Cássio Cancio | Orientadores: Prof. Paulo Meirelles e Arthur Pilone
        \end{minipage}
        \hfill
        \begin{minipage}{0.1\linewidth} % largura da coluna da imagem
            \includegraphics[width=\linewidth]{vertical-simplificada}
        \end{minipage}
    }

    \posterbox[footerbox]{name=footerbox, above=bottom, column=1, span=12}{
    }

    \posterbox[adjusted title = Contexto]
    {name=context, below=titlebox, column=1, span=6, rowspan=0.7}{
        \begin{itemize}
            \item A adoção de ferramentas de IA generativa para a engenharia\\ de software (ES) é uma das grandes tendências da indústria.

            \item A \emph{Stack Overflow Developer Survey} aponta, desde 2023, o crescimento da adoção de ferramentas de IA para este fim na comunidade de desenvolvedores profissionais.
        \end{itemize}
    }

    \posterbox[adjusted title = Objetivo]
    {name=objective, below=titlebox, column=7, span=6}{
        \begin{itemize}


            \item Levantar dados sobre os impactos da IA generativa na \\ produção de software
            \item Analisar a evolução da percepção dos desenvolvedores sobre \\ o uso destas ferramentas, de acordo com diversas pesquisas e \\ relatórios do mercado
            \item Comparar resultados de diferentes fontes da literatura \\ cinza e formal
        \end{itemize}
    }

    %%%%%%%% Quatro colunas com "equal height group" %%%%%%%%

    \posterbox[adjusted title = Uso de IA,
        smallmargins]
    {name=so, below=context, column=1, span=6, rowspan=1.6}{

    \begin{center}
        \small
        \textbf{Uso de IA no desenvolvimento de \emph{software} em 2025 por programadores profissionais (Stack Overflow)}
        \includegraphics[width=11cm]{grafico-uso.png}
    \end{center}

 
    Segundo previsões da \emph{Gartner}:
        \begin{itemize}
            \item Em 2028, 90\% das empresas de \emph{software} usarão \emph{AI code} \\ \emph{assistants};
            \item A IA passará de 5\% (2024) para 40\% (2027) em participação no \\ ciclo de desenvolvimento de \emph{software};
            \item 25\% dos \emph{bugs} em produção virão de código de IA não revisado.
        \end{itemize}
            
            



    }

    \posterbox[adjusted title = Casos de uso da IA,
        smallmargins]
    {name=riscos, below=objective, column=7, span=6}{
        Segundo uma pesquisa da Elsevier, os desenvolvedores entrevistados preferem delegar tarefas de que gostam \\ menos de fazer para ferramentas de IA e manter o controle das\\ mais prazerosas:
\vspace{1cm}


    \begin{tabular}{p{11cm}p{7cm}p{7cm}}
                Atividade  & Delegaria & Gosta de fazer \\
                \toprule

                Escrita de testes & \cellcolor{green!10} 70\% & \cellcolor{red!10} 30\%\\
                Documentação técnica & \cellcolor{green!10} 66\% & \cellcolor{red!10} 26\%\\
                Criar novas \emph{features}& \cellcolor{red!10} 27\% & \cellcolor{green!10} 86\%\\
                
            \end{tabular}

\vspace{1cm}
            
    A Stack Overflow Developer Survey 2024 perguntou qual uso os usuários fazem das ferramentas de IA.
    
    \begin{itemize}
        \item 82\% utiliza para escrever código;
        \item 40,1\% utiliza para produzir documentação;
        \item 27,2\% utiliza para testar código.
    \end{itemize}
    }


    %%%%%%%% Duas colunas %%%%%%%%

    % Como temos 2 caixas à esquerda e uma caixa à direita, não podemos
    % simplesmente usar "equal height group" aqui, então definimos
    % manualmente a altura das caixas de maneira que as duas colunas
    % tenham o mesmo tamanho.

    %%% Esquerda
    \posterbox[adjusted title = Análise de Qualidade de Código (GitClear)]
    {name=gitclearText, below=riscos,
        column=7, span=6, rowspan=1.51}{

        \textbf{1. Queda na refatoração}
        \begin{itemize}
            \item A proporção de linhas movidas, forte indicador de refatoração,\\ caiu de 24,1\% (2020) para 9,5\% (2024).
        \end{itemize}

        \textbf{2. Aumento das duplicações de código:}
        \begin{itemize}
            \item Em 2024, pela primeira vez, a porcentagem de linhas copiadas (12,3\%) superou as linhas movidas (9,5\%).
            \item A ocorrência de commits com blocos de código duplicados cresceu de 0,7\% (2020) para 6,66\% (2024).
        \end{itemize}

        \textbf{3. Aumento do Retrabalho:}
        \begin{itemize}
            \item O \textit{Churn} (proporção do código revertido em menos de 2\\ semanas) cresceu para 5,7\%.
        \end{itemize}

        \textbf{4. Redução da Durabilidade do Código:}
        \begin{itemize}
            \item A proporção de linhas que em menos de duas semanas foram alteradas aumentou de 60,4\% em 2020 para 69,7\% em 2024
            \item Já as linhas que precisaram de alterações apenas entre 4 semanas e um ano, diminuiu de 24,7\% para 16,9\% no mesmo \\ período.
        \end{itemize}
    }

    \posterbox[adjusted title = Tabela das Métricas (GitClear)]
    {name=gitclearVisual, below=so,
        column=1, span=6, rowspan=1.25}{

        A GitClear produziu uma análise de 211 milhões de linhas de \\ código alteradas entre 2020 e 2024. O código tem origem em\\  milhares de repositórios de empresas e repositórios de código aberto. 

        \vspace{0.5cm}

        Os resultados das métricas foram:
        
        \vspace{0.5cm}

        \begin{center}    
            \footnotesize
            \begin{tabular}{p{2cm}p{4.2cm}p{4.2cm}p{4.2cm}p{4.2cm}p{4.2cm}}
                Ano  & Adicionado & Movido& Cópia  & Substituído & Churn \\
                \toprule
                2020 & 39,2\%     & 24,1\%& 8,3\%  & 2,9\%       & 3,1\% \\


	2021 & 39,5\% & 24,8\% & 8,4\% & 3,4\% & 3,3\% \\

    2022 & 40,9\% & \cellcolor{red!10} 20,5\%& 9,4\%  & 3,7\%       & 3,3\% \\
	
    2023 & \cellcolor{green!15} 42,3\% & \cellcolor{red!20} 15,8\% & \cellcolor{red!10} 10,6\% & \cellcolor{green!10} 3,6\% & \cellcolor{red!10} 4,5\% \\
    
    2024 & \cellcolor{green!30} 46,2\% & \cellcolor{red!40} 9,5\% & \cellcolor{red!20} 12,3\% & \cellcolor{green!20} 4,2\% & \cellcolor{red!20} 5,7\% \\
                
            \end{tabular}
                    
   
        \end{center}
            }

    %%% Direita
    \posterbox[adjusted title = Referências, smallmargins]
    {name=bibbox, between=gitclearVisual and footerbox, column=1, span=8}{
        \footnotesize
        \begin{itemize}
            \item \textbf{[GitClear 2025]} Harding, W. "AI Copilot Code Quality: Evaluating 2024's Increased Defect Rate".
            \item \textbf{[Gartner 2024]} "Principais Tendências Tecnológicas Estratégicas para 2025: IA Agêntica".
            \item \textbf{[Gartner MQ]} "Magic Quadrant for AI Code Assistants", 2024.
            \item \textbf{[Stack Overflow]} "Developer Survey 2023/2024/2025".
            \item \textbf{[Elsevier 2025]} Sergeyuk, A. et al. "Using AI-based coding assistants in practice".
            \item \textbf{[ANSSI/BSI]} "AI Coding Assistants - Security Recommendations", 2024.
        \end{itemize}
    }

    \posterbox[adjusted title = Site do TCC, smallmargins]
    {between=gitclearVisual and footerbox, column=9, span=4}{

        \begin{center}
            \qrcode[height=2.5in]{https://cassiocancio.github.io/TCC-BCC/}

            \vspace{1cm}

            Acesso à monografia completa.
        \end{center}

    }
\end{tcbposter}

\end{document}
