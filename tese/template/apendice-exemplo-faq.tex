%!TeX root=../tese.tex
%("dica" para o editor de texto: este arquivo é parte de um documento maior)
% para saber mais: https://tex.stackexchange.com/q/78101

\chapter{Perguntas frequentes sobre o modelo}

\begin{itemize}

\item \textbf{Não consigo decorar tantos comandos!}\\
Use a colinha que é distribuída juntamente com este modelo (\url{gitlab.com/ccsl-usp/modelo-latex/raw/main/pre-compilados/colinha.pdf?inline=false}).

\item \textbf{Estou tendo problemas com caracteres acentuados.}\\
Versões modernas de \LaTeX{} usam UTF-8, mas arquivos antigos podem usar outras codificações (como ISO-8859-1, também conhecido como latin1 ou Windows-1252). Nesses casos, use \textsf{\textbackslash{}usepackage[latin1]\{inputenc\}} no preâmbulo do documento. Você também pode representar os caracteres acentuados usando comandos \LaTeX{}: \textsf{\textbackslash\textquotesingle{}a} para á, \textsf{\textbackslash{}c\{c\}} para cedilha etc., independentemente da codificação usada no texto\footnote{Você pode consultar os comandos desse tipo mais comuns em \url{en.wikibooks.org/wiki/LaTeX/Special_Characters}. Observe que a dica sobre o pingo do i \emph{não} é mais válida atualmente; basta usar \textsf{\textbackslash\textquotesingle{}i}.}.

\item \textbf{É possível resumir o nome das seções/capítulos que aparece no topo das páginas e no sumário?}\\
Sim, usando a sintaxe \textsf{\textbackslash{}section[mini-titulo]\{titulo enorme\}}. Isso é especialmente útil nas legendas (\textit{captions}\index{Legendas}) das figuras e tabelas, que muitas vezes são demasiadamente longas para a lista de figuras/tabelas.

\item \textbf{Existe algum programa para gerenciar referências em formato bibtex?}\\
Sim, há vários. Uma opção bem comum é o JabRef; outra é usar Zotero\index{Zotero} ou Mendeley\index{Mendeley} e exportar os dados deles no formato .bib.

\item \textbf{Posso usar pacotes \LaTeX{} adicionais aos sugeridos?}\\
Com certeza! Você pode modificar os arquivos o quanto desejar, o modelo serve só como uma ajuda inicial para o seu trabalho.

\end{itemize}
